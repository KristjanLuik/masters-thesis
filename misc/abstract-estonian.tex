Metsavarade jätkusuutlikuks majandamiseks on hädavajalik metsaraiete tuvastamise täpsus ja ajakohasus. Traditsioonilised statistilised meetodid Sentinel‑2 multispektraalsete piltide töötlemisel on osutunud tõhusaks, kuid kannatavad sageli piiratud ruumilise eraldusvõime ja mahuka käsitöö tõttu. Käesolev uurimus käsitleb isejuhitud Vision Transformer’i peaahelat DINOv2 väheste õppeandmete põhist semantilist segmentatsiooni lageraie sündmuste tuvastamiseks.

Esmalt loodi andmestik, mis koosneb 100 lageraie polügoonist ja nende ümbritsevatest metsapiirkondadest. Nende põhjal ehitati programm, mis töötleb ja analüüsib metsateatisi, et luua geomeetrilised maskid, mis eristavad okas- ja lehtpuudega alasid. Lisaks sellele laeb programm alla Sentinel‑2 taseme 2A satelliidi pildid, et luua andmestik, mis sisaldab nii lageraie maske kui ka nende ümbritsevaid metsapiirkondi.

Tulemuste hindamiseks võrdleme DINOv2‑põhist raamistikku Random Foresti ja U‑Neti segmentatsioonimudelitega, kasutades peamiste kvaliteedimõõdikutena IoU‑d ja F1 skoori. Avatud lõpptulemite analüüs võimaldab hinnata tuvastustäpsuse, märgendamistõhususe ja mudeli robustsuse paranemist erinevates metsakoostistes. Edasistes uuringutes käsitletakse ajarealise analüüsi ja täiustatud pilvekatte maskimise strateegiaid.

% No need to change this
Lõputöö on kirjutatud \langEst~keeles ning sisaldab teksti \calculatepages leheküljel, 
\total{totalchapters} peatükki\ifthenelse{\equal{\totvalue{figure}}{0}}{}{% If no figures, do nothing
, \total{figure} \ifnum\totvalue{figure}=1 joonis\else joonist\fi%
}\ifthenelse{\equal{\totvalue{table}}{0}}{}{% If no tables, do nothing
, \total{table} \ifnum\totvalue{table}=1 tabel\else tabelit\fi%
}.