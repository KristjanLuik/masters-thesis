\begin{longtable}{p{3cm}p{10cm}}  % Begins a longtable environment. The 'p{3cm}' and 'p{10cm}' specify column widths.
Ülelennu sagedus & (\textit{Revisit time}) ajavahemik, mis jääb mingi kindla
piirkonna satelliidivaatluste vahele\\
Lainepikkus & (\textit{Band}) lainepikkuste vahemik elektromagnetkiirguse spektris \\
peaahel & (\textit{Backbone}) mudeli peamine arhitektuur, mis on eelnevalt treenitud
ja millele on lisatud täiendavad kihid, et saavutada soovitud ülesanne\\
R2 & R-ruut (\textit{R-squared}) regressioonimudeli täpsuse mõõdik, mis näitab
mudeli selgitusvõimet andmete variatsioonis\\
IoU & Ühenduse indeks (\textit{Intersection over Union}) mõõdik, mis hindab
mudeli täpsust, võrreldes ennustatud ja tegelikke tulemusi\\
F1 skoor & F1 skoor (\textit{F1 score}) mõõdik, mis ühendab täpsuse ja
meeldetuletuse ja annab tasakaalustatud hinnangu mudeli jõudlusele\\
NDVI & Taimede indeksi (\textit{Normalized Difference Vegetation Index}) mõõdik, mis
hindab taimekatte tihedust ja elujõudlust, arvutatakse punase ja lähedase
infrapunase lainepikkuse vahekorra põhjal\\
Jääkühik & Jääkühikud (\textit{residual blocks}) on närvivõrgu arhitektuurimuster, kus konvolutsioonikihide jadale lisatakse sisendi ja väljundi vahe (residuaal) skip-ühenduse kaudu, et hõlbustada sügavate võrkude treenimist ja vältida gradientide kadu. \\
DICE & DICE (\textit{Dice coefficient}) on mõõdik, mis hindab segmentatsiooni \\
\end{longtable}
\addtocounter{table}{-1} % Decreases the table counter by 1 so that it doesn't increment the table number in the document