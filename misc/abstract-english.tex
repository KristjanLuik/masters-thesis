Monitoring and accurately detecting forest logging activities is essential for sustainable forest management and environmental conservation. Traditional statistical approaches, such as Random Forest models applied to Sentinel-2 imagery, have shown promise but still suffer from limited spatial precision and require extensive manual post‑processing. In this thesis, we explore the efficacy of a self‑supervised Vision Transformer backbone, DINOv2, for few‑shot semantic segmentation of logging events in Sentinel‑2 multispectral images.

A smaller dataset comprising 100 clear‑cut polygons and their adjacent forest environs was constructed by integrating publicly available metsateatis records from the Estonian Forest Registry with Sentinel‑2 Level‑2A surface reflectance tiles. Initial geometric masks were refined through manual delineation and K‑Means clustering to differentiate coniferous and deciduous strata. The pretrained DINOv2 model was subsequently fine‑tuned on this dataset, utilizing the 10 m spatial resolution spectral bands alongside derived vegetation indices to enable pixel‑level discrimination of logging areas.

To evaluate performance, we compare the DINOv2‑based framework against benchmark Random Forest and U‑Net segmentation models, using Intersection over Union (IoU) and F1 score as primary metrics. This open-ended analysis will assess relative improvements in detection accuracy, annotation efficiency, and robustness to varying forest compositions. Further investigation will address temporal sequence incorporation and advanced cloud‑masking strategies.


% No need to change this
The thesis is in \langEng~and contains \calculatepages pages of text, 
\total{totalchapters} chapters\ifthenelse{\equal{\totvalue{figure}}{0}}{}{% If no figures, do nothing
, \total{figure} \ifnum\totvalue{figure}=1 figure\else figures\fi%
}\ifthenelse{\equal{\totvalue{table}}{0}}{}{% If no tables, do nothing
, \total{table} \ifnum\totvalue{table}=1 table\else tables\fi%
}.