Selles peatükis analüüsime eelmises peatükis kirjeldatud katsete tulemusi, et mõista, kuidas erinevad mudelikonfiguratsioonid ja treeningstrateegiad mõjutavad metsastunud ja lageraielõikude tuvastamise täpsust satelliitpiltidelt. Arutatakse ka tulemuste statistilist olulisust ning võrreldakse erinevate lähenemiste efektiivsust.
\section{Tulemuste võrdlus}
\textbf{Baasjoon}

Eksperimendi käigus testiti süstemaatiliselt erinevaid hüperparameetreid, et leida optimaalne konfiguratsioon antud ülesande jaoks. 
Esitatud tulemuste analüüs näitas, et parimad mudelid
saavutasid märkimisväärselt kõrgeid Dice'i skoore, ulatudes kuni \textasciitilde
0.984. See on eriti üllatav, arvestades andmestiku koostamise väikest valimit.

Parimaks osutunud mudeli konfiguratsioon oli järgmine:
\begin{itemize}
  \item \textbf{Arhitektuur:} Unet
  \item \textbf{Enkooder:} ResNet50
  \item \textbf{Enkooderi kaalud:} ImageNet (eelkoolitatud)
  \item \textbf{Optimeerija:} AdamW
  \item \textbf{Õpisamm:} \textasciitilde 1.09e-4
\end{itemize}

Ka teised kombinatsioonid,
näiteks DeepLabV3 koos ResNet50-ga, saavutasid kõrgeid tulemusi (Dice > 0.97).
Osaliselt on tipptulemused visualiseeritud ka lisatud joonisel \ref{fig:segmentation_results}.

\textbf{Kõrgete skooride põhjused}
Kõrgeid tulemusi väikesel andmestikul võib seletada mitme teguriga.
Siirdõpe (\textit{Transfer Learning}): ImageNet andmestikul eelkoolitatud enkooderite
kasutamine on tõenäoliselt peamine edu võti. Eelnevalt treenitud mudelid omavad
juba võimekust tuvastada üldiseid visuaalseid mustreid (nt servad, tekstuurid),
mida saab efektiivselt kohandada spetsiifilisele metsanduse segmenteerimise
ülesandele. See vähendab oluliselt vajamineva treeningandmestiku mahtu. Peaaegu
kõik parimad tulemused saavutati just imagenet kaaludega. 
Kuigi valideerimistulemused on kõrged, on väikse andmestiku kasutamine riskantne, sest alati peab arvestama ülesobitamise (\textit{overfitting}) ohuga. Samas näitab analüüs, et paljudel juhtudel valideerimis- ja treeningkahju
(val\_loss, train\_loss) vähenesid sünkroonis, mis viitab sellele, et mudelid
suutsid siiski valideerimisandmetele edukalt üldistuda ega õppinud
treeningandmeid lihtsalt pähe.

\begin{figure}[H]
    \centering
    \includegraphics[width=0.8\textwidth]{figures/top3_dice.png}
    \caption{Parimad DICE tulemused erinevate mudelite vahel.}
    \label{fig:segmentation_results}
\end{figure}

Samas individuaalsed klassi tulemused (nt metsastumine vs lageraie) Dice skoorid näitasid suuresti, et mudel ei suuda
täpselt eristada metsaga kaetud alasid ja lageraielõike. Kõrge skoor tuleneb suuresti tagatausta suurest osakaalust pildil. Joonisel \ref{fig:sidebyside_forest_bg} on näha et ülejäänud tausta keskmine Dice skoor on 0.98, mis on väga kõrge. Aga klassipõhised tulemused näitavad, et metsade ja lageraie alade eristamine on keeruline. Mudelite katsed ka näitavada, et ka parimate mudelite puhul ei suuda nad metsa ja lageraiet eristada.

\begin{figure}[H] % Placement specifier: h-here, t-top, b-bottom
    \centering
    \begin{subfigure}[b]{0.8\textwidth} % Set width for side-by-side arrangement
        \includegraphics[width=\textwidth]{figures/tulemused/dice_per_class_y_forest.png} % Image path and full width in subfigure
        \caption{} % Leave caption empty to automatically label this subfigure as (a)
        \label{fig:dice_per_class_y_forest}
    \end{subfigure}
    % Second subfigure
    \begin{subfigure}[b]{0.8\textwidth} % Set width for side-by-side arrangement
        \includegraphics[width=\textwidth]{figures/tulemused/dice_per_class_background.png} % Path and full width in subfigure
        \caption{} % Leave caption empty to automatically label this subfigure as (b)
        \label{fig:dice_per_class_background}
    \end{subfigure}
    
    \caption{Tagatausta ja noore metsa segmentatsiooni Dice tulemused üle katsete} % Main caption for the whole figure environment
    \label{fig:sidebyside_forest_bg} 
\end{figure}




\textbf{Dinov2 eksperimendid}

Käesolevas uurimuses viidi läbi eksperimente, mille eesmärk oli hinnata DINOv2
raamistikul põhinevate masinõppemudelite efektiivsust satelliidipiltide
segmenteerimisel metsanduslikus kontekstis. Katsetati erinevaid Vision
Transformer (ViT) arhitektuure (nt ViT-B/14, ViT-G/14, ViT-L/14, ViT-S/14) koos
erinevate segmenteerimispeadega (nt LinearHead, SimpleHead, FPNHead,
UPerNetHead). Mudelite jõudlust hinnati keskmise Dice'i koefitsiendi, keskmise
IoU (Intersection over Union) ja keskmise täpsuse (Mean Accuracy) alusel,
jälgides neid mõõdikuid kuni 800 epohhi vältel. Treeningutingimustes varieeriti
hüperparameetreid, nagu õppimismäär (nt 1e-5, 5e-5, 1e-4), partii suurus (1, 2,
4) ning rakendati nii külmutamist (Freeze) kui ka peenhäälestamist (FineTune).

\bigskip
\begin{longtable}{llll}
    \textbf{Konfiguratsioon} & \textbf{Mean Dice} & \textbf{Mean IoU} & \textbf{Mean Täpsus} \\
    \hline
    ViTB14 LinearHead LR5e-5 BS2 E5 Freeze & 0.18 & 0.16 & 0.26 \\
    ViTB14 SimpleHead LR1e-5 BS4 E5 FineTune & 0.200 & 0.26 & 0.35 \\
    ViTG14 FPNHead LR1e-5 BS4 E5 FineTune & 0.258 & 0.28 & 0.33 \\
    ViTG14 SimpleHead LR1e-4 BS4 E5 Freeze & 0.22 & 0.275 & 0.350 \\
    ViTL14 UPerNetHead LR1e-5 BS1 E3 Freeze & 0.160 & 0.270 & 0.34 \\
    ViTS14 FPNHead LR1e-5 BS4 E5 FineTune & 0.23 & 0.25 & 0.38 \\
    \hline
    \caption{DINOv2 eksperimendi tulemused}
    \label{tab:dinov2_results}
\end{longtable}
\bigskip

Eksperimentide tulemused näitavad, et DINOv2-põhiste mudelite jõudlus satelliidipiltide segmenteerimisel sõltub oluliselt närvivõrgu selgroo suurusest, segmenteerimispea keerukusest ja treeningustrateegiast. Suuremad närvivõrkude selgrood (nt ViTG14) ja mitme skaala tunnuseid töötlevad pead (nt FPNHead) andsid parimaid tulemusi, saavutades keskmise IoU kuni 0,28 ja täpsuse kuni 0,38. Peenhäälestamine parandas jõudlust võrreldes külmutatud mudelitega, kuid pikaajalistes treeningutes (500–800 epohhi) ilmnes ebastabiilsus, mis viitab ületreenimisele või ebapiisavale regulatsioonile.

Soovitused edasiseks uurimistööks hõlmavad varajase peatamise ja õppimismäära ajakava rakendamist stabiilsuse suurendamiseks, samuti täiendavate arhitektuuride (nt hübriidmudelite) testimist jõudluse lae ületamiseks. Praegused tulemused pakuvad väärtuslikku alust metsanduslike satelliidipiltide segmenteerimise optimeerimiseks.

\textbf{SAMv2 eksperimendid}

Sarnaselt DINOv2 katsetele viidi läbi ka eksperimendid Segment Anything Model (SAMv2) raamistiku kasutamisel. SAMv2 on tuntud oma võime poolest tegeleda erinevate segmentatsiooniülesannetega, sealhulgas satelliidipiltide analüüsiga. Katsetes uuriti erinevaid hüperparameetreid, sealhulgas õppimismäära, partii suurust ja treeninguetappe, et leida parim konfiguratsioon metsanduse kontekstis.

Tabelis on seitse eksperimenti, mis erinevad mudeli variandi, epohhide arvu,
õppimiskiiruse ja segmenteerimispea tüübi poolest. Kõik eksperimendid kasutavad
501 epohhi. Parima konfiguratsiooni leidmiseks võrdleme IoU ja Dice väärtusi, kuna need on peamised näitajad mudeli jõudluse hindamisel. Tabelist selgub, et kõrgeimad tulemused on järgmised:

\begin{itemize}
    \item Kõrgeim IoU:  0.350205 (konfiguratsioon: large, õppimiskiirus $1 \times 10^{-5}$, Multi-Layer pea).
    \item Kõrgeim Dice:  0.262514 (sama konfiguratsioon: large, õppimiskiirus $1 \times 10^{-5}$, Multi-Layer pea).
\end{itemize}

\begin{longtable}{ccccccc}
    \hline
    Variant & Epochs & Learning Rate & Head Type & IoU & Dice \\
    \hline
        tiny &  501 & $1 \times 10^{-4}$ & Simple & 0.229138 & 0.150820\\
        tiny &  501 & $1 \times 10^{-4}$ & Multi-Layer & 0.256366 & 0.178527 \\
        large &  501 & $1 \times 10^{-4}$ & ASPP & 0.327914 & 0.246620  \\
        large &  501 & $1 \times 10^{-4}$ & U-Net Style & 0.289425 & 0.205382  \\
        large &  501 & $1 \times 10^{-5}$ & Multi-Layer & 0.350205 & 0.262514 \\
        base\_plus & 501 & $1 \times 10^{-4}$ & Multi-Layer & 0.279433 & 0.195139  \\
        large &  501 & $1 \times 10^{-4}$ & Simple & 0.242327 & 0.163633 \\
    \hline
\caption{Samv2 eksperimendi tulemused}
\label{tab:samv2_results}
\end{longtable}

Järgnevas joonises on välja toodud SAMv2 eksperimendi tulemused eksperimendi lõikes ja jaotadud klassideks.

\begin{figure}[H]
    \centering
    \includegraphics[width=0.8\textwidth]{figures/tulemused/Sam_experimentide_tulemused_klassid.png}
    \caption{SAMv2 eksperimendi tulemused klasside lõikes. }
    \label{fig:sam_result}
\end{figure}

Kasutades eelmises osas treenitud SAMv2 mudelit, genereerisime segmentatsioonikaardid:
\begin{figure}[H]
    \centering
    \includegraphics[width=0.8\textwidth]{figures/tulemused/prediction_3403037_TCI.png}
    \caption{SAMv2 pareima mudeli segmentatsioonitulemused}
    \label{fig:sam_best_result}
\end{figure}

Metsateatistel põhinev segmentatsiooni tulemus:
\begin{figure}[H]
    \centering
    \includegraphics[width=0.6\textwidth]{figures/tulemused/ground_truth_mask_3403037.png}
    \caption{Metsateatistel põhinev segmentatsiooni tulemus}
    \label{fig:actual_mask}
\end{figure}



\textbf{Analüüsi kokkuvõte}

Kokkuvõttes näitavad eksperimendid, et DINOv2 ja SAMv2 raamistike kasutamine satelliidipiltide segmenteerimisel metsanduse kontekstis võib anda häid tulemusi, kuid nõuab hoolikat hüperparameetrite valikut ja treeningstrateegiate rakendamist. Parimad tulemused saavutati suuremate närvivõrkude selgroogude ja keerukamate segmenteerimispeadega, mis suudavad efektiivselt töödelda erinevaid visuaalseid mustreid. Tulevikus võiks uurida ka hübriidmudelite kasutamist, mis kombineerivad erinevate arhitektuuride eeliseid.

\section{Edasiarendus ja täiustamine}
Ehkki käesolev töö demonstreeris isejuhitud Vision Transformer'i põhisele DINO
v2 mudelile ja SAM v2 väheste õppeandmete lähenemisele tugineva segmentatsiooni
edukust Sentinel-2 piltidel, on olemas mitmeid võimalusi täiendusteks ja
metoodilisteks parendusteks:

\textbf{Andmestiku suurendamine}
Käesolevas uurimuses kasutati 15 lageraie polügooni. Tulevastes töödes võiks
andmestikku oluliselt laiendada, kaasates rohkem lageraie juhtumeid ja
erinevaid metsatüüpe. Samuti käesolevas töös ei võetud arvesse hooajalisi
muutusi, seega võiks lisada andmeid erinevatest aastaaegadest ja ilmastikutingimustest,
et suurendada mudeli üldistusvõimet. Näiteks võiks isegi luua mitu mudelit, eri aastaaegade ja ilmastikutingimuste jaoks, et saavutada paremad tulemused.

\textbf{Super-resolutsiooni rakendamine} 
Kuna Sentinel-2 ruumiline lahutusvõime on
piiratud (10 m/piks), võib väikeste lageraie fragmentide täpsemaks tuvastamiseks
kasutada süvaõppel põhinevaid super-resolutsiooni mudeleid (nt ESRGAN, RCAN).
Eeltehtud super-resolutsiooni mudelid suudaksid tõsta sisendi detailsust,
võimaldades segmenteerimisel paremini eristada kitsaid raiemustreid ja segatud
taimestiku.

\textbf{Rohkemate spektraalribade kaasamine} Uuringus keskendusime 10 m ja 20 m
riba­komplektile (B02--B08), kuid Sentinel-2 andmestikul on ka 60 m
lahutusvõimega B01, B09, B10 ribad. Nende ja teiste indeksite (nt NDMI, EVI,
SAVI) kombineerimine võib lisada eristavust pilvede, märgade alade ja erinevate
puuliikide vahel. Lisaks tasub uurida Sentinel-1 S-banda radariandmete liitmist,
et parandada pilvekatte all segmentatsiooni ja tuvastada aluspinnamuutusi.

\textbf{Kõrgema resolutsiooniga satelliidiandmete integreerimine} Kommertssatelliidid
nagu WorldView-3 (0,3 m piks), Pleiades (0,5 m) ja PlanetScope (3 m) pakuvad
oluliselt paremat ruumilist lahutust. Nende kõrglahutusega andmete
mitmespektriline analüüs (nt PAN-sharpening) võib toetada treeningandmete
loomist ja mudeli peenhäälestust, võimaldades tuvastada ja valideerida
väiksemaid raietegevuse detaile.

\textbf{Ajarealist analüüsi süvendamine} Käesolev töö käsitles iga lageraie juhtumit
eraldi hetke põhiselt. Järgmistes etappides võiks integreerida
ajaseeria-segmendatsiooni (nt ConvLSTM, Temporal CNN), et modelleerida metsa
arengut ja raiemustreid ajas. See võimaldaks tuvastada hooajalisi ja
aastate-üleseid trende ning eristada ajutisi muutusi (nt varisemine, põua- või
tulekahju-kahjustused) tegelikest lageraietest.

\textbf{Täiustatud pilve- ja varjumaskimine} Pilve ning varjude eemaldamine piiras osa
piltide kasutamist. Tulevases töös võiks katsetada nn ``cloud shadow''
tuvastamise meetodeid (nt Fmask, Sen2Cor täiendused) või sügavõppel põhinevaid
pilvemaskimudeleid, et taastada katkenud ajaread ja saavutada pidevam andmevoog.

\textbf{Mudeli peenhäälestuse automatiseerimine} Hüperparameetrite käsitsi otsing on
ajamahukas. Võiks rakendada Bayesia optimeerimist või evolutsioonialgoritme (nt
Optuna, Hyperopt) hüpermäärade leidmiseks, samuti kasutada automaatset Data
Augmentation strateegiate valikut (nt AutoAugment), et tugevdada mudeli
üldistusvõimet väheste andmetega.

\textbf{Üldistusvõime ja siirdõpe uutes piirkondades} Järgnevalt on soovitatav hinnata
mudeli ülekantavust erinevates geograafilistes ja metsatüüpides (nt
okaspuumetsad Põhja-Soomes, troopilised metsad Indoneesias). Siirdõpe (transfer
learning) sobiva allmudeliga ja peenhäälestus väheste lokaalsete näidete põhjal
aitaks välja selgitada, millises mahus mudel vajab uut treeningandmestikku
globaalseks rakenduseks.
