Mets katab 51,5\%\footnote{\url{https://www.keskkonnaagentuur.ee/keskkonnaseire-ja-analuusid/mets}} Eesti pindalast ning metsaraiete järelvalve on üks Keskkonnaagentuuri seireülesannetest. Metsa raiumiseks peab Eestis kehtiva korra järgi metsaomanik esitama Keskkonnaametile vastava metsateatise, mille heakskiitmisel on metsaomanikul vastava raie tegemiseks 2 aastat aega. Keskkonnaagentuuri ülesannete hulka kuulub raiete tuvastamine ja kontroll, kas need raied on tehtud kehtiva metsateatise alusel.  

Praegusel hetkel kasutatakse lisaks raiete ortofotodelt ja satelliidipiltidelt inimeksperdi abil tuvastamisele ka Eestis 2020.-ndal aastal Keskkonnaagentuuri ja Tartu Ülikooli koostöös väljatöötatud masinõppe mudelit, mis raie tuvastamiseks satelliidipiltidelt kasutab otsustusmetsa (Random forest) algoritmi \cite{TartuUlikooliTeadlased2020}. Selle mudeli esmased tulemused olid paljulubavad, aga jooksvas kasutamise pole see ikkagi rahuldavaid tulemusi andnud ja Keskkonnaagentuuri töötajad on sunnitud siiski pigem tuginema ekspertide arvamusele.

Antud magistritöö põhieesmärgiks on võrrelda kaasaegseid masinõppe meetodeid sarnases rakenduses ja tuua välja täpseim mudel, mis suudaks rahuldava täpsusega tuvastada metsaraiet satelliidipiltidelt. Mida aeg edasi seda rohkem on riigid hakanud mõistma kui tähtis on metsamajandus, metsade säilitamine ja hoidmine. Tehnoloogia pideva arenguga on hakatud otsima viise kuidas riik või kogukond saaksid paremat ülevaadet suurtest metsaga kaetud aladest. Metsa seireks kasutatakse peamiselt mehitamata õhusõidukeid (Unmanned Aerial Vechicles), maapealseid sensoreid, satelliidipildi töötlust ja vabatahtlike kaasavaid rakendusi (Crowdsourcing Applications) \cite{cheungPerimeterDefense42015}.

Euroopa Liidu kaugseireprogramm Copernicus võimaldab Eesti riigil koguda satelliidipilte andmekeskusesse Esthub \cite{maa-ametRiiklikSatelliidiandmeteKeskus}. Lisaks muule infole, mida hallatakse Copernicus-es ja seeläbi ka Esthub-is, on kasutusel informatsioon, mis tuleb erinevatelt Sentineli nime kandvatelt satelliitidelt \cite{InfrastructureOverviewCopernicus}. Kuna Sentinel-2 on juba 2015. aastast töös olnud, sisaldab laia valikut valgusspektreid ning on tiheda korduskülastussagedusega \cite{Sentinel2OverviewScienceDirect}, siis keskendub käesolev magistritöö peamiselt sellele satelliidi tüübile.

Tööd tehes kujunes välja mõistmine, et automaatselt ei ole võimalik saada alusandmeid. Seetõttu tuli võtta kasutusele täielikult käsitsi märgendatud andmed, mille loomine on väga ajamahukas (ca 3$km^2/h$). Seoses väikse andmekogumiga, on magistritöös ka keskendutud andmekesksele tehisintellektile (\textit{data-centric AI}). Andmekeskses tehisintellektis on rõhk süsteemsel tööl andmetega mudeli keerukuse kasvatamise asemel: hinnatakse andmekogumi kvaliteeti, mitmekesisust ja usaldusväärsust.

Andmekogumi loomisel on üheks alam eesmärgiks luua Python programm, mis hõlbustaks satelliidipiltide allalaadimist ja töötlemist. Peale andmete kogumist on viime läbi tänapäevaste masinõppe mudelite võrdluse raiete tuvastamiseks. Raiet hinnatakse piksli põhise täpsusega üle pildi. Hiljuti on tehtud mitmeid uuringuid selles valdkonnas, kus kasutatakse ka otsustusmetsi, XGBoost ja U-Net’il põhinevaid mudeli arhitektuure \cite{isaienkovDeepLearningRegular2021}, \cite{podoprigorovaRecognitionForestDamage2024}. Mõlemas uurimistöös on ka mudelite võrdlus välja toodud, aga need keskenduvad erinevatele suundadele. Esimese puhul ehitatakse mudelid kasutades rohkem pilte läbi aja, et mudel saaks paremini tuvastada muutust. Teise puhul keskendutakse erinevate lainepikkuste kombineerimisele, et tabada muutusi. 

Peale mudelite treenimist samadelt lähteandmetelt on antud magistritöös välja toodud tulemuste mõõtmine. Piksli tasemel täpsuse mõõtmiseks kasutatakse Intersection over Union - kattuvuse hinnang, Dice Coefficient - meetrika mis on põhimõtteliselt segmenteerimise F1 Score \cite{IntersectionUnionIoU}, \cite{UnderstandingDICECOEFFICIENT}. Nende tulemuste abil saab teha võrdluse erinevate tuvastusmudelite vahel, et leida neist täpseim.
% TODO: meetrika mis on põhimõtteliselt segmenteerimise F1 Score- minu arvates kõlab liiga kõnekeelselt magistritöö kohta. kuidagi ümbersõnastada kui saab.

Sellest tulenevalt sai antud magistritöö uurimisküsimusteks:
\begin{itemize}
    \item Kuidas on võimalik luua automatiseerida satelliidipiltide allalaadimist ja töötlemist ning treeningandmete ettevalmistamist?
    \item Kui täpselt on võimalik tuvastada lageraie sündmusi satelliidi piltidelt kasutades selleks arvutinägemise alusmudeleid?
    \item Milline andmekogum on vajalik, et saavutada piisav täpsus lageraie sündmuste tuvastamiseks?
\end{itemize}