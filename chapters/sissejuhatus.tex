Antud magistritöö põhieesmärgiks on võrrelda masinõppe meetodeid ja tuua välja täpseim mudel, mis suudaks tuvastada metsaraiet satelliidipiltidelt. Mida aeg edasi seda rohkem on riigid hakanud mõistma kui tähtis on metsamajandus, metsade säilitamine ja hoidmine. Tehnoloogia pideva arenguga on hakatud otsima viise kuidas riik või kogukond saaksid paremat ülevaadet suurtest metsaga kaetud aladest. Metsa seireks kasutatakse peamiselt mehitamata õhusõidukeid (Unmanned Aerial Vechicles), maapealseid sensoreid, satelliidipildi töötlust ja vabatahtlike kaasavaid rakendusi (Crowdsourcing Applications) \cite{cheungPerimeterDefense42015}.

Praegusel hetkel kasutatakse Eestis mõni aasta tagasi Keskkonnaagentuuri ja Tartu Ülikooli koostöös väljatöötatud statistika mudelit, mis raie tuvastamiseks kasutab suvasalu (Random forest) algoritmi \cite{TartuUlikooliTeadlased2020} satelliidi piltidelt. Selle mudeli esmased tulemused olid paljulubavad, aga peale pikemat kasutamist pole see ikkagi rahuldavaid tulemusi andnud ja mudeli kasutajad on sunnitud siiski manuaalseid viise kasutama.

Euroopa Liidu kaugseireprogramm Copernicus võimaldab Eesti riigil koguda satelliidi pilte andmekeskusesse Esthub \cite{maa-ametRiiklikSatelliidiandmeteKeskus}. Lisaks muule infole, mida hallatakse Copernicus-es ja seeläbi ka Esthub-is, on kasutusel informatsioon, mis tuleb erinevatelt Sentineli nime kandvatelt satelliitidelt \cite{InfrastructureOverviewCopernicus}. Kuna Sentinel-2 on juba 2015. aastast töös olnud, sisaldab laia valikut valgusribasid ning on tiheda korduskülastus sagedusega \cite{Sentinel2OverviewScienceDirect}, siis keskendub käesolev magistritöö peamiselt sellele satelliidi tüübile.

Seoses väikse andmekogumiga, on magistritöös ka keskendutud andmekesksele tehisintellektile (\textit{data-centric AI}). Andmekeskne tehisintellekt rõhutab andmete süsteemset tööd mudeli keerukuse asemel: andmekogumi kvaliteeti, mitmekesisust ja usaldusväärsust käsitletakse kui tähtsaimat osa.

Sellest tulenevalt on üheks alam eesmärgiks luua Python programm, mis hõlbustaks satelliidi piltide allalaadimist ja töötlemist. Peale andmete kogumist on plaan läbi viia tänapäevaste masinõppe mudelite võrdlus raiete tuvastamiseks. Raiet hinnatakse piksli põhise täpsusega üle pildi. Hiljuti on tehtud mitmeid uuringuid selles valdkonnas, kus kasutatakse ka suvasalu, XGBoost ja U-Net’il põhinevaid mudeli arhitektuure \cite{isaienkovDeepLearningRegular2021}, \cite{podoprigorovaRecognitionForestDamage2024}. Mõlemas uurimistöös on ka mudelite võrdlus välja toodud, aga need keskenduvad erinevatele suundadele. Esimese puhul ehitatakse mudelid kasutades rohkem pilte läbi aja, et mudel saaks paremini tuvastada muutust. Teise puhul keskendutakse erinevate lainepikkuste kombineerimisele, et tabada muutusi. 

Peale mudelite treenimist samadelt lähteandmetelt on antud magistritöös välja toodud tulemuste mõõtmine. Piksli tasemel täpsuse mõõtmiseks kasutatakse Intersection over Union - kattuvuse hinnang, Dice Coefficient - meetrika mis on põhimõtteliselt segmenteerimise F1 Score \cite{IntersectionUnionIoU}, \cite{UnderstandingDICECOEFFICIENT}. Nende tulemuste abil saab teha võrdluse erinevate tuvastusmudelite vahel, et leida neist täpseim.
% TODO: meetrika mis on põhimõtteliselt segmenteerimise F1 Score- minu arvates kõlab liiga kõnekeelselt magistritöö kohta. kuidagi ümbersõnastada kui saab.

Sellest tulenevalt sai antud magistritöö uurimisküsimusteks:
\begin{itemize}
    \item Kui täpselt on võimalik tuvastada lageraie sündmusi satelliidi piltidelt kasutades selleks DinoV2 alus mudelit?
    \item Kas on võimalik luua automatiseeritud programm, mis suudab allalaadida ja töödelda satelliidi pilte?
    \item Milline andmekogumik on vajalik, et saavutada kõrge täpsus lageraie sündmuste tuvastamiseks?
\end{itemize}