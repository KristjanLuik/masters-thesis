\section{Andmestiku loomine}
Selles peatükis käsitleme andmestiku loomise protsessi, sealhulgas andmete kogumist, töötlemist ja maskimist. Andmestiku loomine on oluline samm igasuguste andmete analüüsi ja masinõppe projektide puhul. Andmete kvaliteet ja sobivus mõjutavad otseselt mudeli täpsust ja usaldusväärsust. Nagu muudes valdkondades kehtib ka informaatikas pareto printsiip, mille kohaselt 80\% probleemidest tuleneb 20\% põhjustest. Seega on andmestiku loomine ja töötlemine äärmiselt oluline etapp, mis võib määrata kogu projekti edasise käigu.

\subsection{Andmete kogumine}
Metsateatis on dokument, mille kaudu metsaomanik esitab Keskkonnaametile
kavandatavate raietööde või oluliste metsakahjustuste kohta teabe. Keskkonnaamet
kontrollib esitatud teatiste nõuetekohasust ning veendub, et kavandatav raie
vastab kehtivatele õigusaktidele. Metsateatised menetletakse ja säilitatakse
riiklikus metsaregistris. Peale edukat menetlemist või raie töö alustada 10 päeva peale otsust ja kuni 24 kuu jooksul. \cite{MetsateatisJaMetsaregister} Metsateatised on avalikud ja neid saab vaadata riiklikus metsaregistris.

Koostöös Keskkonnaametiga (envir) saime andmed metsateatistest, mis sisaldavad teavet nii metsateatise esitamise kuupäeva, metsateatise menetlemise kuupäeva, metsateatise kehtivuse alguskuupäeva kui ka metsateatise kehtivuse lõppkuupäeva kohta. Kuna riigimetsade teatised on täpsemas seisukorras siis sai võetud need raie teatisedaluseks. Seoses sellega et ühe lõigu peal võib olla väga väike kogus metsa, sai teatiste pärimine ümber ehitatud sedasi, et ühe riigimetsa raie ümber kogutakse peale raie toimumist kokku ka kõik teised piirkonnad millel on teada kas on mets või raie. Piirkonniti pärimine sai teostatud kasutades PostGISi liidestust Posgresi andmebaasiga.

\begin{figure}[hb]
    \centering
    \includegraphics[width=.5\textwidth]{figures/andmestik/er_id_is10124223.png}
    \caption{Näidis lageraie ümbruse päringust saadud ümbrus}
    \label{fig:umbrusexample}
\end{figure}

\begin{itemize}
    \item kust andmeid saab

    \item kuidas andmeid töödelda
    \begin{itemize}
        \item kuidas andmeid puhastada
        \item kuidas andmeid ühendada
        \item kuidas andmeid lõigata
        \item kuidas andmeid venitada
        \item kui andmeid on puudu või kuidas filtreerida
    \end{itemize}

    
\item kuidas andmeid maskida