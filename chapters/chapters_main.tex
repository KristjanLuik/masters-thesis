\zlabel{firstpagetocount}       % DO NOT REMOVE! Used for counting number of pages of main text

% Part Labels Explanation:
% Labels in the form \label{chapter:<part_name>} are used to categorize chapters into specific parts of the thesis structure.
% These labels are essential for calculating the ratio of text dedicated to different thesis sections (e.g., Introduction, Methodology, Results, Discussion, Summary).
% Allowed labels:
% - chapter:introduction -> Introduction (<10% of the text).
% - chapter:method       -> Methodology (<20% of the text).
% - chapter:results      -> Results (30–40% of the text).
% - chapter:discussion   -> Analysis, Discussion, and Conclusions (30–40% of the text).
% - chapter:summary      -> Summary (<0.5 pages).
% How to use: Add a \label{chapter:<part_name>} for each chapter to indicate its part. 
% These labels ensure consistency and allow automated tests to validate the thesis structure.

\chapter{Sissejuhatus}\label{chapter:sissejuhatus} % This command creates a numbered chapter titled "Sissejuhatus" (Estonian for "Introduction"). The \label{chapter:introduction} assigns a label to the chapter, allowing you to reference it later (e.g., \ref{chapter:introduction}).
Mets katab 51,5\%\footnote{\url{https://www.keskkonnaagentuur.ee/keskkonnaseire-ja-analuusid/mets}} Eesti pindalast ning metsaraiete järelvalve on üks Keskkonnaagentuuri seireülesannetest. Metsa raiumiseks peab Eestis kehtiva korra järgi metsaomanik esitama Keskkonnaametile vastava metsateatise, mille heakskiitmisel on metsaomanikul vastava raie tegemiseks 2 aastat aega. Keskkonnaagentuuri ülesannete hulka kuulub raiete tuvastamine ja kontroll, kas need raied on tehtud kehtiva metsateatise alusel.  

Praegusel hetkel kasutatakse lisaks raiete ortofotodelt ja satelliidipiltidelt inimeksperdi abil tuvastamisele ka Eestis 2020.-ndal aastal Keskkonnaagentuuri ja Tartu Ülikooli koostöös väljatöötatud masinõppe mudelit, mis raie tuvastamiseks satelliidipiltidelt kasutab otsustusmetsa (Random forest) algoritmi \cite{TartuUlikooliTeadlased2020}. Selle mudeli esmased tulemused olid paljulubavad, aga praktilises kasutamises pole see ikkagi rahuldavaid tulemusi andnud ja Keskkonnaagentuuri töötajad on sunnitud siiski pigem tuginema ekspertide arvamusele.

Antud magistritöö põhieesmärgiks on võrrelda kaasaegseid masinõppe meetodeid sarnases rakenduses ja tuua välja täpseim mudel, mis suudaks rahuldava täpsusega tuvastada metsaraiet satelliidipiltidelt. Mida aeg edasi seda rohkem on riigid hakanud mõistma kui tähtis on metsamajandus, metsade säilitamine ja hoidmine. Tehnoloogia pideva arenguga on hakatud otsima viise kuidas riik või kogukond saaksid paremat ülevaadet suurtest metsaga kaetud aladest. Metsa seireks kasutatakse peamiselt mehitamata õhusõidukeid (Unmanned Aerial Vechicles), maapealseid sensoreid, satelliidipildi töötlust ja vabatahtlike kaasavaid rakendusi (Crowdsourcing Applications) \cite{cheungPerimeterDefense42015}.

Euroopa Liidu kaugseireprogramm Copernicus võimaldab Eesti riigil koguda satelliidipilte andmekeskusesse Esthub \cite{maa-ametRiiklikSatelliidiandmeteKeskus}. Lisaks muule infole, mida hallatakse Copernicus-es ja seeläbi ka Esthub-is, on kasutusel informatsioon, mis tuleb erinevatelt Sentineli nime kandvatelt satelliitidelt \cite{InfrastructureOverviewCopernicus}. Kuna Sentinel-2 on juba 2015. aastast töös olnud, sisaldab laia valikut valgusspektreid ning on tiheda korduskülastussagedusega \cite{Sentinel2OverviewScienceDirect}, siis keskendub käesolev magistritöö peamiselt sellele satelliidi tüübile.

Tööd tehes kujunes välja mõistmine, et automaatselt ei ole võimalik saada alusandmeid. Seetõttu tuli võtta kasutusele täielikult käsitsi märgendatud andmed, mille loomine on väga ajamahukas (ca 3$km^2/h$). Seoses väikse andmekogumiga, on magistritöös ka keskendutud andmekesksele tehisintellektile (\textit{data-centric AI}). Andmekeskses tehisintellektis on rõhk süsteemsel tööl andmetega mudeli keerukuse kasvatamise asemel: hinnatakse andmekogumi kvaliteeti, mitmekesisust ja usaldusväärsust.

Andmekogumi loomisel on üheks alam eesmärgiks luua Python programm, mis hõlbustaks satelliidipiltide allalaadimist ja töötlemist. Peale andmete kogumist on viime läbi tänapäevaste masinõppe mudelite võrdluse raiete tuvastamiseks. Raiet hinnatakse piksli põhise täpsusega üle pildi. Hiljuti on tehtud mitmeid uuringuid selles valdkonnas, kus kasutatakse ka otsustusmetsi, XGBoost ja U-Net’il põhinevaid mudeli arhitektuure \cite{isaienkovDeepLearningRegular2021}, \cite{podoprigorovaRecognitionForestDamage2024}. Mõlemas uurimistöös on ka mudelite võrdlus välja toodud, aga need keskenduvad erinevatele suundadele. Esimese puhul ehitatakse mudelid kasutades rohkem pilte läbi aja, et mudel saaks paremini tuvastada muutust. Teise puhul keskendutakse erinevate lainepikkuste kombineerimisele, et tabada muutusi. 

Peale mudelite treenimist samadelt lähteandmetelt on antud magistritöös välja toodud tulemuste mõõtmine. Piksli tasemel täpsuse mõõtmiseks kasutatakse Intersection over Union - kattuvuse hinnang, Dice Coefficient - meetrika mis on põhimõtteliselt segmenteerimise F1 Score \cite{IntersectionUnionIoU}, \cite{UnderstandingDICECOEFFICIENT}. Nende tulemuste abil saab teha võrdluse erinevate tuvastusmudelite vahel, et leida neist täpseim.
% TODO: meetrika mis on põhimõtteliselt segmenteerimise F1 Score- minu arvates kõlab liiga kõnekeelselt magistritöö kohta. kuidagi ümbersõnastada kui saab.

Sellest tulenevalt sai antud magistritöö uurimisküsimusteks:
\begin{itemize}
    \item Kuidas on võimalik luua automatiseerida satelliidipiltide allalaadimist ja töötlemist ning treeningandmete ettevalmistamist?
    \item Kui täpselt on võimalik tuvastada lageraie sündmusi satelliidi piltidelt kasutades selleks arvutinägemise alusmudeleid?
    \item Milline andmekogum on vajalik, et saavutada piisav täpsus lageraie sündmuste tuvastamiseks?
\end{itemize} %  This command inserts the content from the file introduction.tex located in the chapters folder.

\chapter{Valdkonna ülevaade}\label{chapter:taust}
\section{Metsandus}
Metsad omavad olulist rolli nii ühiskonna igapäevaelus kui ka planeedi heaolus.
Alates mööblis kasutatavast puidust kuni paberini, millele kirjutame. Lisaks neile
nähtavatele toodetele sisaldavad paljud ravimid, kosmeetika ja pesuvahendid
metsadest saadud kõrvalsaadusi. Rohkem kui 1,6 miljardit inimest sõltub
metsadest toidu ja kütuse saamisest ning umbes 70 miljonit, sealhulgas paljud
põlisrahvad, peavad metsi oma koduks \cite{karsentyUnderlyingCausesRapid2003}. 
Metsad varustavad meid hapnikuga, pakuvad
varjualust, töökohti, puhast vett ja toitu, olles seega inimkonna ellujäämiseks
hädavajalikud. Kuna nii paljude inimeste elu sõltub metsadest, on metsade saatus
otseselt seotud ka meie endi tulevikuga. \cite{WWFImportanceForests} 

\section{Copernicus ja EstHub}
Copernicus on üks osa Euroopa kosmoseprogrammist (EUS), mis tegeleb planeedi jälgimisega. Copernicus programmi raames, lisaks maa pealse info kogumisele, on loodud mitmeid satelliite, mis koguvad informatsiooni kosomosest. See info on kõigile kättesaadav tasuta. Selle programmiga seotud satellite kutsutakse \textbf{Sentineliks}. \cite{CopernicusCopernicus}


EstHub on Eesti riiklik satelliitandmete keskus, mis kogub ja integreerib
mitmekesiseid georuumilisi andmeid automatiseeritud protsesside kaudu.
Andmekogumine hõlmab kõrge resolutsiooniga satelliitkaadrite allalaadimist ja
standardiseerimist erinevatest allikatest. EstHubi eesmärk on koguda kokku sateliidi andmed mis katavad Eesti territooriumi. \cite{maa-ametNationalSatelliteData}

\begin{figure}[H]
    \centering
    \includegraphics[width=.3\textwidth]{figures/datahubEU.drawio.png}
    \caption{Sateliidi andmete liikumine andmekeskuste vahel}
    \label{fig:esthubliiklus}
\end{figure}


\subsection{Sentinel}
Sentinel-1 on radaripõhine satelliit, mis võimaldab jälgida maapinna vajumist,
struktuuride kahjustusi ning looduskatastroofe nagu maavärinad ja maalihked. Samuti on
see ideaalne mere- ja Arktika seireks, sealhulgas laevade jälgimiseks ning
naftareostuse tuvastamiseks. \cite{S1Applications}

Sentinel-2 missioon koosneb kahest identsest satelliidist, Sentinel-2B
(käivitatud 2017) ja Sentinel-2C (käivitatud 2024), mis töötavad koos, et
pakkuda kõrge eraldusvõimega multispektraalseid pilte Maa pindadest,
rannikualadest ja siseveekogudest iga viie päeva järel. Need andmed toetavad
rakendusi põllumajanduses, metsanduses ja maakatte klassifitseerimisel. \cite{S2Applications}

Sentinel-3 on Euroopa Maa seire satelliitmissioon, mille eesmärk on mõõta
merepinna topograafiat, mere ja maa pinnatemperatuure ning ookeani ja maa
pinnavärvi suure täpsusega. Neid andmeid kasutatakse ookeani prognoosisüsteemides,
keskkonnaseires ja kliimaseires. \cite{S3Mission}

Sentinel-5P on esimene Copernicuse missioon, mis on pühendatud atmosfääri
seirele. See kannab tipptasemel \textbf{Tropomi} instrumenti, mis kaardistab mitmeid
gaase nagu lämmastikdioksiid, osoon, formaldehüüd, vääveldioksiid, metaan,
vingugaas ja aerosoolid - kõik need mõjutavad meie hingatavat õhku, tervist ja
kliimat. \cite{S5PApplications}
\subsection{Lainepikkuste spekter}
Spektriribad on satelliitandmete analüüsimisel üliolulised, sest need
võimaldavad eristada maapinna erinevaid omadusi, lähtudes elektromagnetilise
spektri konkreetsetest lainepikkustest. Näiteks Sentinel-2 MSI instrumendi 13
spektririba hõlmavad nähtavat valgust, lähedast infrapunat ja lühilaine
infrapunat, võimaldades detailset maastiku klassifitseerimist, sealhulgas
metsade, veekogude ja muu loodusliku keskkonna eristamist. Iga ribaga seondub
kindel lainepikkuse vahemik, mida spetsiifiliste filtrite abil eraldatakse. \cite{S2Mission}
\bigskip

\begin{longtable}{llll}
    \hline
    Riba & Resolutsioon & Kasutus                          \\ 
    \hline
    B01  & 60$m \mkern3mu px^{-1}$        & Aerosool                         \\
    B02  & 10$m \mkern3mu px^{-1}$        & Sinine                           \\
    B03  & 10$m \mkern3mu px^{-1}$        & Roheline                         \\
    B04  & 10$m \mkern3mu px^{-1}$        & Punane                           \\
    B05  & 20$m \mkern3mu px^{-1}$        & Vegetatsiooni klassifitseerimine \\
    B06  & 20$m \mkern3mu px^{-1}$        & Vegetatsiooni klassifitseerimine \\
    B07  & 20$m \mkern3mu px^{-1}$        & Vegetatsiooni klassifitseerimine \\
    B08  & 10$m \mkern3mu px^{-1}$        & Lähiinfrapunariba on hea rannajoonte ja biomassisisalduse kaardistamiseks \\
    B8A  & 20$m \mkern3mu px^{-1}$        & Kitsam lähedane infrapunane  \\
    B09  & 60$m \mkern3mu px^{-1}$        & Veeaur tuvastus                       \\
    B10  & 60$m \mkern3mu px^{-1}$        & Pilvede tuvastus                      \\
    B11  & 20$m \mkern3mu px^{-1}$        & Lühilaine infrapunane 1      \\
    B12  & 20$m \mkern3mu px^{-1}$        & Lühilaine infrapunane 2      \\
         &              &                    &                              \\ \hline
    \caption{Sentinel-2 MSI spektriribad ja nende kasutusvaldkonnad}
    \label{tab:s2bands}
\end{longtable}

\subsection{Koordinaatsüsteemid ja CRS}
Koordinaatsüsteem on meetod, mille abil määratletakse ja kirjeldatakse punktide
asukohti maastikul, kasutades koordinaate. Selles kontekstis eristatakse kahte
tüüpi: geograafilised koordinaatsüsteemid, mis kasutavad laiuse ja pikkuse
väärtusi, ning projekteeritud koordinaatsüsteemid, mis teisendavad
geograafilised koordinaadid lameda kaardi koordinaatideks, kasutades
matemaatilisi projektsioone. CRS ehk koordinaatide viite süsteem määratleb
reeglid ja parameetrid, mille alusel need koordinaadid seonduvad reaalse
maastikuga. \cite{8CoordinateReference}

\section{Masinõppe meetodite kasutus kaugseires}
Käsitletava teemaga seotud, kuid teiste piirkondade põhjal loodud, uurimistööde analüüsimisel prooviti välja selgitada, milliseid meetodeid on üldiselt kaugseires kasutatud, sealjuures kas on ehitatud mudeleid nullist või kasutatud
valmis mudeleid. Samuti oli oluliseks eesmärgiks, välja selgitada, kas nendel juhtudel on kasutatud süvaõpet või mitte. Lisaks sooviti teada saada, mis satelliidi andmeid on varasemalt kasutatud ja kas eri lainepikkuste sidumine on andnud paremaid tulemusi.

Ukraina teadlaste poolt koostatud 2021. aastal välja antud artiklis \glqq Deep Learning for Regular Change
Detection in Ukrainian Forest Ecosystem With Sentinel-2\grqq{} kasutati Copernicus Sentinel-2 satelliidi pilte, mis sisaldasid kõrge resolutsiooniga (10 m) värvi- ja spektrikanaleid, sealhulgas NDVI ja NDMI indekseid, võimaldades jälgida metsamuutusi kuni 5-päevaste intervallidega. Andmekogum loodi käsitsi Kharkivi piirkonnas, kasutades mitut järjestikust pilti ja põhjalikku märgistust, et tagada täpne deforestsatsioonipiirkondade kaardistamine. Uurijad rakendasid süvaõppe meetodeid, kasutades mitut U-Neti varianti (näiteks UNet-diff, UNet-CH, UNet2D, UNet3D, Siamese U-Netid ja UNet-LSTM), et hinnata nii ajast sõltuvaid kui ka ühekordseid lähenemisviise. Eraldi rõhutati piltidevahelise erinevuse kasutamise eeliseid, mis parandas segmentatsioonitulemusi ning tõstis Dice ja F1 skoore. Lisaks ilmnesid uuringus olulised nüansid, nagu pilvekatte, hooajaliste muutuste ning geograafiliste lahknevuste mõju, mis nõudsid täiendavat andmete eeltöötlust. Huvitav on, et kuigi kõik mudelid näitasid potentsiaali, saavutavad UNet-diff ja UNet-CH kõige kõrgema täpsuse, pakkudes seeläbi tõenduspõhiseid lahendusi metsakatteta ala muutuste regulaarseks jälgimiseks. \cite{isaienkovDeepLearningRegular2021}

Uus-Meremaal läbi viidud ja 2024. aastal välja antud uurimistöös \glqq Developing a forest description from remote sensing: Insights from
New Zealand\grqq{} kasutati kõrglahutusega lennufotosid ning regionaalseid ALS-andmeid radiata männi metsade täpseks kaardistamiseks Uus-Meremaal. Analüüs tugines sügavõppepõhisel semantilise segmentatsiooni mudelil, mis kasutab DeepLabv3+ arhitektuuri koos ResNext-101 peaahelana (backbone), saavutades IoU väärtused 0,94, täpsuse 0,96 ja meeldetuletuse 0,98. Keeruliseks osutus aga noorte istikute tuvastamine, mille puhul "juvenile" klass (noored istutatud metsapiirkonnad) liideti "radiata" (küpsemad männi alad) klassiga. Lisaks sügavõppemudelile kasutati mitmemuutujalisi regressioonimudeleid metsade keskmise kõrguse, kogumahte ja vanuse hindamiseks, saavutades kõrged R2 väärtused. \cite{pearseDevelopingForestDescription2025}

\chapter{Lahendus}\label{chapter:lahendus}
\section{Töövahendid}


\textbf{Python}

Python on üldotstarbeline programmeerimiskeel, mida kasutatakse
laialdaselt andmeteaduse, masinõppe ja ruumiandmete analüüsi ülesanneteks
oma lihtsuse ja mitmekülgsuse tõttu. Käesolevs töös kasutatakse Pythonit
andmete töötlemiseks, mudelite treenimiseks ja tulemuste analüüsimiseks.



\textbf{Jupyter Notebookid}\nopagebreak[4]

Jupyter Notebookid pakuvad interaktiivset keskkonda, kus
saab koodi kirjutada, käivitada ja dokumenteerida ühes kohas. Need võimaldavad
dünaamilist andmeanalüüsi ja tulemuste visuaalset esitlust, muutes
uurimisprotsessi läbipaistvaks ja korduvaks. Käesolevas töös kasutatakse
Jupyter Notebooke peamiselt katsetuste tegemiseks ja tulemuste
visualiseerimiseks.



\textbf{Pandas}\nopagebreak[4]

Pandas on andmetöötluse teek, mis pakub paindlikke ja
efektiivseid andmestruktuure tabelipõhise andmetöötluse jaoks. See lihtsustab
andmete puhastamist, analüüsi ja manipuleerimist. Käesolevas töös
kasutatakse Pandast andmete lugemiseks, töötlemiseks ja analüüsimiseks.


\textbf{GeoPandas}\nopagebreak[4]

GeoPandas laiendab Pandase võimalusi, lisades tuge georuumilistele
andmetele. See võimaldab lugeda, analüüsida ja visualiseerida
ruumiandmeid ning teostada geomeetrilisi operatsioone nagu lõikumine ja
ühendamine.


\textbf{Rasterio}\nopagebreak[4]

Rasterio on Pythoni teek, mis keskendub rasterandmete lugemisele
ja töötlemisele tuginedes GDAL-ile. See võimaldab ruumiandmete analüüsi
ning laseb rasterfailidega töötada efektiivselt ja intuitiivselt. Käesolevas töös oli Rasterio peamiseks tööriist rasterandmete lugemiseks ja töötlemiseks, sealhulgas satelliidipiltide koostamiseks ja maskide genereerimiseks.

\textbf{QGIS}\nopagebreak[4]

QGIS on tasuta ja avatud lähtekoodiga töölaua GIS-tarkvara, mis võimaldab
kasutajatel andmeid visuaalselt analüüsida, redigeerida ja kaardistada. See
toetab mitmeid andmeformaate ja pakub laialdasi geoprotsessimise võimalusi,
olles populaarne nii akadeemilises kui ka professionaalses keskkonnas. Käesolevas
töös kasutatakse QGIS-i peamiselt andmete visualiseerimiseks ja
analüüsimiseks, et mõista ruumiandmete struktuuri ja omadusi.


\textbf{PostGIS}\nopagebreak[4]

PostGIS on PostgreSQL andmebaasi laiendus, mis lisab ruumiandmete töötlemise funktsionaalsuse. See võimaldab keerukaid ruumi operatsioone
ja on oluline tööriist suurte ruumiandmete kogude haldamisel ning
analüüsil. Käesolevas töös autor ise otseselt PostGIS-i ei kasutanud, kuid Keskkonnaagentuuri andmebaas, kust andmed saadi, on PostGIS-i baasil üles ehitatud. Seega on see oluline komponent andmete haldamisel ja töötlemisel.

\textbf{Riistvara}\nopagebreak[4]

Uurimistöö läbiviimisel kasutati ülikooli AI-labori ressursse. Labor koosneb ühest pea-sõlmest, mis haldab teisi masinaid, ning alamsõlmedest, mis teostavad töid. Autor kasutas ai-lab-07 sõlme CPU-intensiivsete ülesannete, nagu andmekogumi koostamine, piltide töötlemine ja kompressioon, ning mitte sügav närvivõrkude mudelite treenimiseks nagu \textit{Random Foresti}. Samas süvaõppe eksperimentide teostamiseks kasutati ai-lab-04 sõlme, mille GPU ja mälumaht võimaldasid keerukamate mudelite treenimist. Selline ressursside jaotus aitas töövoogu optimeerida ja tagada tööde sujuva teostuse vastavalt konkreetsetele arvutusvajadustele.
\begin{longtable}{lllll}
    \hline
    Sõlm & Protsessor & Mälu & GPU & GPU mälu                         \\ 
    \hline
    ai-lab-07 & 3960X 24-cores/48-threads & 128 GB & NVidia 2080Ti & 11 GB    \\
    ai-lab-04 & 3970X 32-cores/64-threads & 128 GB & NVidia 3090 & 24 GB      \\
    &              &                    &                              \\ \hline
    \caption{Kasutatud riistvara}
    \label{tab:hardwareused}
\end{longtable}

\section{Andmestiku loomine}
Käesolevas osas käsitletakse andmestiku loomise protsessi, sealhulgas andmete kogumist, töötlemist ja maskimist. Andmestiku loomine on oluline samm igasuguste andmete analüüsimisel ja seda ka masinõppe projektide puhul. Andmete kvaliteet ja sobivus mõjutavad otseselt mudeli täpsust ja usaldusväärsust. Nagu muudes valdkondades kehtib ka informaatikas Pareto printsiip, mille kohaselt 80\% probleemidest tuleneb 20\% põhjustest. Seega on andmestiku loomine ja töötlemine äärmiselt oluline etapp, mis võib määrata kogu projekti edasise käigu.

\subsection{Raie piirkonna andmete kogumine}
Metsateatis on dokument, mille kaudu metsaomanik esitab Keskkonnaametile
kavandatavate raietööde või oluliste metsakahjustuste kohta teabe. Keskkonnaamet
kontrollib esitatud teatiste nõuetekohasust ning veendub, et kavandatav raie
vastab kehtivatele õigusaktidele. Metsateatised menetletakse ja säilitatakse
riiklikus metsaregistris. Peale edukat menetlemist võib raietöödega alustada 10 päeva peale otsust ja kuni 24 kuu jooksul. \cite{MetsateatisJaMetsaregister} Metsateatised on avalikud ja neid saab vaadata riiklikus metsaregistris.

Metsade inventeerimise ja registrisse kandmise protsess algab metsaeraldiste
täpse kaardistamisega, kasutades L‑EST97 ristkoordinaatide süsteemi, Eesti
põhikaarti, katastriüksuse plaane ning vajadusel kaugseire andmeid eraldiste
piiritlemiseks ja võimalike situatsioonielementide täpsustamiseks. Kaardistamise
tulemusena koostatakse geoinfosüsteemi metsaeraldiste kiht, kus iga eraldis on
nummerdatud ning selle pindala, arvutatuna piiripunktide koordinaatide alusel, 
esitatakse hektarites vähemalt kümnendkohani ning täpsusega 10 meetrit --- see
loob aluse usaldusväärsele pindalaarvestusele ja edaspidistele
takseerimistoimingutele. \cite{MetsaKorraldamiseJuhend}

Koostöös Keskkonnaagentuuriga saadi andmed metsateatistest, mis sisaldavad teavet nii metsateatise esitamise kuupäeva, metsateatise menetlemise kuupäeva, metsateatise kehtivuse alguskuupäeva kui ka metsateatise kehtivuse lõppkuupäeva kohta. Kuna riigimetsade teatised on täpsemas seisukorras, siis võeti need raieteatised selle uurimustöö aluseks. Seoses sellega et ühe lõigu peal võib olla väga väike kogus metsa, sai teatiste pärimine ümber ehitatud sedasi, et ühe metsa raie ümber kogutakse peale raie toimumist kokku ka kõik teiste raiete raadiuses asuvad piirkonnad, millel on teada, kas on mets või raieala. Piirkonniti pärimine sai teostatud kasutades PostGISi liidest Postgresi andmebaasiga. Iga raie sisaldab ka endas geomeetria veergu, mis esitab polügooni kujul selle asukohta. 

Polügoon ehk hulknurk on geomeetriline kujund, mis määratleb kindla ala, ühendades üksteisega
punktid, et moodustada suletud piirjoon. Andmetöötluse ja ruumiandmete analüüsi
kontekstis kasutatakse polügoone, et täpselt määratleda geograafilisi alasid. \cite{WhatLocationPolygon}


\begin{figure}[H]
    \centering
    \hspace{-0.5cm}\subfloat[]{{\includegraphics[width=0.5\textwidth]{figures/andmestik/er_id_is_3308099.png}
    	}}
    %\begin{subfigure}[b]{0.5\textwidth} % Set width for side-by-side arrangement
    %    \centering
    %    \includegraphics[width=0.5\textwidth]{figures/andmestik/er_id_is_3308099.png}
    %    \caption{}
    %    \label{fig:umbrusexample}
    %\end{subfigure}
    %% Second subfigure
    \subfloat[]{{\includegraphics[width=0.5\textwidth]{figures/andmestik/lr_3308099_TCI.png}
}}

    %\begin{subfigure}{0.5\textwidth} 
    %   \centering
    %    \includegraphics[width=0.5\textwidth]{figures/andmestik/lr_3308099_TCI.png}
    %    \caption{} % Leave caption empty to automatically label this subfigure as (b)
    %    \label{fig:satellite_example}
    %\end{subfigure}

    \caption{Näidis ühe lageraie päringust saadud ümbrusest (a) ja satelliitpildist (b).}
    \label{fig:sidebyside_teatis_sat_img}
\end{figure}

Magistritöö peamiseks uurimisküsimuseks on, kas ja kuidas on võimalik kasutada väheste näidete (\textit{Few-Shot}) põhist alusmudelit. Väikese valimi puhul on aga vaja eriti täpseid andmeid millelt õppida. Nagu eelnevalt mainitud siis metsaregistrist saadud andmed ei ole piisavalt täpsed, et neid otse kasutada alusmudeli treenimisel metsaraie tuvastamiseks, mis on illustreeritud Joonisel \ref{fig:sidebyside_teatis_sat_img}. Esiteks on küsimus, kui kaua pärast raieteatise kehtima hakkamist metsa raiuti, kui üldse raiuti, ja teiseks esineb raieid, mis ei püsi rangelt metsajaotiste piirides. Lisaks paistab raiutud mets raiutud metsana erineva perioodi vältelt sõltuvalt raie aastaajast -- suvisel ajal muutub raiutud ala kiiremini roheliseks.

Seetõttu lisandusid uurimistöösse andmete täiendava töötlemise ja maskimise etapid, kus andmed käiakse käsitsi läbi, kasutades registri andmeid maski põhjana. Ajalise piirangu tõttu pidi tegema alamvalimi. Et saada Eesti metsade kohta üldisemaid näiteid kasutati selleks KMeans klasterdamise meetodit, et jagada metsad omakorda kahte erinevasse klassi okas- ja lehtpuud. Eesmärgiks oli koguda kokku 100 raiet ja nende ümbrust, et luua piisavalt andmeid, mille pealt mudelit treenida. Kui peaks juhtuma et mõni pilt pole kõlbulik, siis valitakse samast alamvalimist uus lageraie piirkond. Valitud alad on toodud Joonisel \ref{fig:kmeans}. 

\begin{figure}[H]
    \centering
    \includegraphics[width=.9\textwidth]{figures/andmestik/kmeansmap.png}
    \caption{KMeans klasterdamise tulemus, 100 raie ümbrust leht- või okaspuudega}
    \label{fig:kmeans}
\end{figure}


Peale selle etapi lõppu, kus on loodud 100 raie ümbrust polügoonidena. Edasi kirjeldatakse lahti sammud, välja toodud Joonisel \ref{fig:terveflow}, et saada nendega seonduvad satelliitpildid.

Töös kasutatud andmestik koosneb 15 lageraie piirkonnast. Alljärgnevalt on tabel mis toob välja lageraie ID ja jäädvustus kuupäeva.
\bigskip


\begin{longtable}{ll}
    \hline
    \textbf{Lageraie ID} & \textbf{Jäädvustus kuupäev} \\
    \hline
        226703.0 & 2023/08/04 \\
        3464763.0 & 2022/10/13 \\
        3310419.0 & 2020/10/11 \\
        3542301.0 & 2023/11/13 \\
        3307283.0 & 2020/09/16 \\
        204935.0 & 2022/09/26 \\
        3543143.0 & 2023/11/07 \\
        3536617.0 & 2023/08/28 \\
        3468479.0 & 2022/10/04 \\
        3453239.0 & 2022/08/23 \\
        \hline
\end{longtable}


Kuna andmete päring ja töötlemine nõudis suurt arvutus- ja salvestusressurssi,
algas töö ühe üksiku lageraie piirkonna ja selle vahetu ümbruse salvestamisega
ühte .shp-vormingus vektorfaili. Edasiste testide käigus selgus, et optimaalne
lähenemine hõlmab 10--15 lageraie ala ning nende ümbruste sõltumatut päringut ja
salvestamist eraldi shape-failidena. Selle põhjal viidi läbi järgnev töölõikude
jadastik:
\begin{enumerate}[topsep=1pt,itemsep=1ex,partopsep=1ex,parsep=1ex]

\item \textbf{Shape-failide loomine ja kokkuliitmine}\newline
Iga lageraie piirkonna ja selle vahetu
ümbruse (buffer) geomeetria päriti riiklikust puistu andmebaasist ja salvestati
iseseisvatesse shapefile'idesse. Selle lähenemise eeliseks oli võimalus
vajalikke alasid hõlpsalt lisada või eemaldada ilma terve andmestiku
ümberlaadimiseta. Paremaks haldamiseks ja analüütiliseks töötluseks liideti kõik
eraldi genereeritud shapefile'id  tervikusse, mis võimaldas
vektorandmete ühtlustatud töötlemist järgmistes etappides.

\item \textbf{Piltide päring Sentinel-2 andmebaasist} \newline
Iga shape-failiga määratletud lageraie
ala serva genereeriti puhvrina ristkülikukujuline ``kast'' (bounding box), mis
täielikult kataks huvipakkuva piirkonna. Selle kasti geomeetria alusel loodi
päring Sentinel-2 metadatebaasist, eesmärgiga leida piirkonnale vastav optiline
satelliitpilt. Päringule lisati järgmised piirangud:

\begin{enumerate}

\item \textbf{Ajapiirang}: pildistamise kuupäev peab jääma lageraie teadmisest alates kuni 40
päeva jooksul, tagamaks muutuste jälgitavuse tuvastamise võimalikult värskel
materjalil.

\item \textbf{Pilvekatte piirang}: pilt valiti vastavalt väikseimale keskmisele pilvekattele,
et maksimeerida nähtavust ja andmete kvaliteeti.
\end{enumerate}

Kui sobiliku satelliitpilti nimetatud tingimustel ei leitud,
laiendati ajavahemikku ning tõsteti lubatavat pilvkatte protsenti kuni eelnevalt
määratud lävendini. Sel juhul, kui sobivat pildistust ikkagi ei tuvastatud,
alustati protsess uuesti järgmise lageraie piirkonnaga samas klastris, et tagada
piisav andmepunktide hulk analüüsi teostamiseks.

\item \textbf{Andmete allalaadimine ja eeltöötlus} \newline
 Sobiva Sentinel-2 toote puhul laeti
vastav .SAFE kataloog alla. Sentineli toode sisaldab üheskoos nii optilisi riba-
(band) kui ka metadatastruktuure, mis on vajalikud edasiseks detailseks
analüüsiks. Eriti olulised on:

\begin{enumerate}
\item \textbf{Spektraalsed ribad} (sh paindliku dünaamikaga B02--B12 ja veespetsiifilised B8,
B8A), mis võimaldavad eristada metsaobjekte ja maapinnanähtusi.

\item \textbf{Pilve- ja lumeindeksiriba (B10)}, mis lihtsustab pilvisuse automaatset
tuvastamist ning tagab eeltöötlemisel täpsema piirkondliku pilvete kaardistamise.
\end{enumerate}

\item \textbf{Atmosfääriline korrektsioon ja lõikamine} \newline
Kõigilt valitud optilistelt ribadelt
viidi läbi atmosfääriline korrektsioon, kasutades Sentinel-2 töötlemise
standardalgoritmi (Sen2Cor või QGIS'i lahtise lähtekoodiga moodul), et eemaldada
atmosfäärist tulenevad häired (nt aerosoolid, gaasiline neeldumine). Pärast
korrektsiooni genereeriti rasterfailide (level-2A) baasil täpsed TIFF-vormingus
pildid, kus iga kanal (band) on salvestatud eraldi kihina, säilitades
kooskõlastatud ruumilist ja spektraalset täpsust.

\item \textbf{Huvipiirkondade täpne väljavõtmine} \newline
Lõpuks lõigati korrektsiooni läbinud
piltidelt välja eelnevalt genereeritud kastide ulatuses täpsed rasterfragmendid,
et töödelda vaid huvipakkuvaid alasid. Selline meetod tagab, et iga lageraie
objekt ja selle ümbrus on rasterandmes täpselt isoleeritud, mis lihtsustab
edasist masinõppe- ja analüüsitöötlust. Lõppfailid salvestati
georefereeritud GeoTIFF-idena, mis toetavad mitmetasandilisi kanalivaateid
ja on integreeritavad GIS-keskkondadesse.

\end{enumerate}
Käesolev töövoog võimaldas optimeerida andmekäitluse ressursikasutust, tagada
piltide kõrge kvaliteet ning kindlustada, et iga lageraie piirkonnast on saadud
piisav hulk informatsiooni edaspidiseks statistiliseks ja mudelipõhisteks
analüüsideks.


\begin{figure}[H]
    \centering
    \includegraphics[width=.8\textwidth]{figures/andmestik/andmete_voog.drawio.png}
    \caption{Andmestiku loomise töövoog}
    \label{fig:terveflow}
\end{figure}


\subsection{Raie piirkonna maskide loomine}
Selles peatükis käsitletakse andmete maskimist, kuna metsaregistrist saadud andmed pole piisavalt täpsed, et neid otse masinõppes kasutada, siis sai loodud sammud, et andmeid täpsemaks muuta. Terviklik dokument on töö lisade hulgas.

Eesmärk oli luua dokument, mis aitaks kõigil kasutajatel luua vektormaske kasutades selleks enne mainitud QGISi tarkvara. Selleks et anda parem hinnang pildil olevatele aladele on kasutusel võetud kihid. Kaks peamist kihti on:
\begin{itemize}[topsep=1pt,itemsep=1ex,partopsep=1ex,parsep=1ex]
    \item \textbf{Metsateatised} --- andmed, mis on saadud riiklikust metsaregistrist ja sisaldavad teavet metsateatise esitamise kuupäeva, metsateatise menetlemise kuupäeva, metsateatise kehtivuse alguskuupäeva ja metsateatise kehtivuse lõppkuupäeva kohta. Selle info abil saab kindlaks teha, kas üldse on vaja märgendada antud piirkonda.
    \item \textbf{Ortofotod} --- on kõrge eraldusvõimega õhust tehtud fotod, mis on georefereeritud ja millel on täpselt määratletud koordinaadid. Ortofotod võimaldavad täpset analüüsi ja hindamist, et tuvastada metsade seisundit ja muudatusi. Ortofotosid tehakse harvemini kui näiteks Sentinel-2 satelliidipilte ja mitte alati üle terve Eesti, kuid need on vajalikud, et saada täpsemat teavet metsade seisundi kohta. Ortofotod on saadaval erinevates spektraalsetes ribades, sealhulgas nähtavas ja infrapunases spektris, mis võimaldab analüüsida erinevaid maapinna omadusi. Käesolevas töös kasutati nii RGB kui ka CIR-NIR ortofotosid. 
\end{itemize}

Teiseks on dokumendis toodud välja juhtnöörid ja soovitused, mida peaks järgima
maskide loomisel. Järgnevalt on välja toodud mõned näidised, mis illustreerivad ortofotode ja Sentinel-2 piltide erinevusi, seega on tähtis, et orotfotosid kautatakse juhtnööridena aga mitte põhitõena. Masinõppes kasutatavate andmete kvaliteet antud uurimistöös mängib suurt rolli, seega kui maskide loomise käigus tekkis kahtlusi klassi tuvastamises, siis jäeti need alad alati välja.

Joonisel \ref{fig:sent_mask_naidis}, Joonisel \ref{fig:orto_mask_naidis} ja Joonisel \ref{fig:orto_nir_mask_naidis} on välja toodud näidised, mis illustreerivad Sentinel-2 saadud pilti ja selle sama piirkonna ortofotosid. NGR värvi tõlgendused on toodud välja varasemas Peatükis \ref{chapter:taust} Tabelis \ref{tab:NGR_kasutus}. Oranžiga on välja toodud piirkond, mille Sentinel-2 sateliidipildi järgi peaks märkima lageraieks, RGB ortofoto samuti viitab sellele. NGR pildil on ka näha tsüaan tooni, mis viitab sellele, et seal peaks olema mulda, ehk suure tõenäosusega võib seal olla lageraie. Siinjuhul peab meeles pidama ka, et uurimistöös kasutatud ortofotode ja Sentinel-2 piltide vahel on ajaline erinevus ja olgugi, et ortofotol on selgesti oranžis kastis kaks lageraiet, siis Sentinel-2 pildi järgi saab märkida ainult ühe neist lageraieks. Metsa näidis on toodud välja rohelise kastiga, millel on kõigil kolmel pildil suuresti ühtlane kattuvus. Eraldi on piltidel kollase kastiga välja toodud heinamaa piirkond. Heinamaad ja põllumaad on alad, mis satelliitpildilt ja RGB ortofoto võivad põgusal vaatamisel tunduda lageraie alana, kuna seal puid ei kasva, aga kui vaadata lähemalt NGR pilti, siis on näha, et seal on vähem mullale viitavat tsüaan tooni, millest võib järeldada, et seal võiks hoopis põld asuda.

\begin{figure}[H]
    \centering
    \includegraphics[width=.6\textwidth]{figures/andmestik/tuvastamis_sent2_naidis.drawio.png}
    \caption{Sentinel-2 RGB näidis}
    \label{fig:sent_mask_naidis}
\end{figure}

\begin{figure}[H]
    \centering
    \includegraphics[width=.6\textwidth]{figures/andmestik/tuvastamis_orto_naidis.drawio.png}
    \caption{RGB ortofoto näidis}
    \label{fig:orto_mask_naidis}
\end{figure}

\begin{figure}[H]
    \centering
    \includegraphics[width=.6\textwidth]{figures/andmestik/tuvastamis_orto_nir_naidis.drawio.png}
    \caption{CIR-NGR ortofoto näidis}
    \label{fig:orto_nir_mask_naidis}
\end{figure}


\section{Alusmudeli ülevaade}
Alustemudelid (\textit{Foundation models}) on suuremahulistel andmekogudel
ennastjuhendavalt treenitud sügavad närvivõrgud, mis toimivad üldotstarbelise
baasina mitmesuguste masinõppe ülesannete lahendamiseks. Erinevalt
traditsioonilistest mudelitest, mis on välja töötatud konkreetse ülesande jaoks
ja nõuavad eraldi treeningut, on alusmudelid eelnevalt ettevalmistatud laia
valiku ülesannete sooritamiseks --- alates loomulikust keele töötlemisest ja
tekstigeneratsioonist kuni pildiklassifitseerimise ja vastuste genereerimiseni
--- ilma täiendava märgendatud õppematerjalita. Nende mudelite
kohanemisvõime tuleneb nii suurest parameetrite hulgast kui ka enesekontrollil
põhinevast õppestrateegiast, mis võimaldab neid hõlpsasti peenhäälestada
konkreetsete rakenduste jaoks. Võimalus keskenduda mudeli peenhäälestusele ja mitte nullist treenimisele omakorda vähendab oluliselt arendusaega ja arvutiressursside vajadust.
\cite{WhatAreFoundation}

\subsection{DINO v2 võrdlus teiste mudelitega}
Käesolev peatükk analüüsib DINO v2 mudeli võrdlust teiste tipptasemel mudelitega, eriti semantilise segmentimise kontekstis, ning põhjendab selle valikut metsaalade ja metsaraie tuvastamiseks satelliidipiltidelt.

Ühes uuringus võrreldi DINO v2 mudelit mitmete teiste segmenteerimismeetoditega
 geoloogiliste kompuutertomograafia (KT) skaneeringute analüüsil. Hinnati Otsu
 lävendamist, klastritehnikaid (K-keskmised, fuzzy C-means), juhendatud
 masinõppe meetodit (Random Forest) ja süvaõppe mudeleid (UNet, ResNet152 ja
 DINOv2). Tulemused näitasid, et eelnevalt treenitud DINO v2 demonstreeris
 tugevat jõudlust kivimipiltide klassifitseerimisel isegi siis, kui
 KT-skaneeringud ei kuulunud selle esialgsesse treeningandmestikku.
 Segmenteerimise ülesannetes ületas eriti LoRA abil peenhäälestatud DINO v2
 teisi meetodeid mitmeklassiliste ülesannete puhul, isegi piiratud andmetega.
 Visuaalne kontroll näitas, et DINO v2 poolt genereeritud segmenteerimismärgid
 olid sageli täpsemad kui algsed sihtmärgid. \cite{DINOv2RocksGeological}

Teises uuringus hinnati DINO v2 jõudlust vasaku koja segmenteerimisel MRI
 piltidelt. Eelnevalt treenitud DINO v2 saavutas Dice'i sarnasuse koefitsiendi
 (DSC)  87.1 \%. Võrdluses nullist treenitud U-Net
 mudeliga näitasid tulemused, et iseseisev õpe DINO v2 abil võib oluliselt
 parandada meditsiinilise pildisegmentimise tõhusust ja täpsust, vähendades
 vajadust suurte märgistatud andmestike järele. \cite{kunduAssessingPerformanceDINOv22024}

Lisaks on Panopticon , DINO v2 raamistikule ehitatud mudel, saavutanud
 tipptasemel tulemusi GEO-Bench võrdlustestis maakatte klassifitseerimisel,
 ületades teisi sensoragnostilisi mudeleid ning näidates konkurentsivõimet ka
 kindlatele sensoritele kohandatud mudelitega. See demonstreerib DINO v2
 arhitektuuri tugevust ja kohanemisvõimet erinevate kaugseire andmetüüpidega. \cite{PanopticonAdvancingAnySensor}

Ühes teises töös võrreldi DINO (DINO v2 eelkäija) laiendust DINO-MC teiste
iseseisvalt õppivate meetoditega kaugseire ülesannetes. Lineaarse testimise
käigus saavutas DINO-MC 2.56\% kõrgema täpsuse väiksema eelnevalt treenitud
andmestikuga kui SeCo. Peenhäälestamisel kahel kaugseire klassifitseerimise ja
muutuste tuvastamise ülesandel ületas DINO-MC nii DINO kui ka SeCo mudeleid. \cite{ExtendingGloballocalView}

Ülaltoodud näited illustreerivad DINO v2 mudeli tugevust ja mitmekülgsust
erinevates segmentimise ülesannetes, sealhulgas geoloogiliste, meditsiiniliste
ja kaugseire piltide analüüsil. Mudeli võime saavutada konkurentsivõimelisi
tulemusi ilma spetsiifilise peenhäälestuseta ning selle parem üldistusvõime
muudavad selle sobivaks valikuks metsaalade ja metsaraie tuvastamiseks
satelliidipiltidelt.
 \subsection{DINO v2}
\textbf{DINO v2} on Meta AI poolt loodud isejuhendatud (self-supervised) mudelite kogum,
mille eesmärk on õppida üldotstarbelisi visuaalseid omadusi ilma märgendatud
andmeteta. Mudel põhineb Vision Transformer (ViT) arhitektuuril, mille erinevad
variandid (nt ViT-S/14, ViT-B/14, ViT-L/14 ja ViT-g/14) on eelõpetatud suurel,
mitmekesise sisuga ja kureeritud pildikogumil. Mudeli struktuuri põhjaks on
õpetaja--õpilase skeem, kus õpilasmudeli parameetreid koheldakse tavalise ViT‑võrguna, aga õpetajamudeli
kaale uuendatakse õpilase kaalude eksponentsiaalse libiseva keskmise kaudu.
Treeningprotsessi stabiliseerimiseks ja tunnusruumi hajutamiseks on lisatud
Kozachenko--Leonenko (KoLeo) regulaarija, mis soodustab tunnuste ühtlast jaotust.
Õppetöö lõpus suurendatakse sisendpiltide resolutsiooni
ajutiselt 518\(\times \)518 pikslile, et parandada piksli tasemel
ülesannete, näiteks semantilise segmenteerimise ja objektituvastuse ennustuse
täpsust. Praktikas saavutab DINO v2 tänu optimeeritud
FlashAttention‑i ja PyTorch Full‑Sharded Data Parallel (FSDP) meetodile kuni
kahekordse kiiruse ning vajab kuni kolm korda vähem mälumahtu, võrreldes varasemate SSL‑mudelitega. \cite{oquabDINOv2LearningRobust2024}

\section{Seoste leidmine}
Selleks et tõestada antud magistritöö eesmärki segmenteerida lageraie ja metsa alasi suurte visioonimudelitega, klusterdatakse DinoV2 väljundid, et leida, kas need alad on eristatavad. Selleks on vajalik eksperdiga valideerida ühe lageraie piirkonna alad, et kindel olla, et need vastavad tõele satelliidipildilt.

Joonisel \ref{fig:näidisSadeliidiPilt} on näidispilt eksperimendist, mille käigus püüti välja selgitada, kas mudelid suudavad eristada lageraie ja metsa alasid satelliidipiltidelt. Antud Sentinel-2 andmebaasist saadud pilt sai valitud esiteks sellepärast, et see on hea nähtavusega, võimalikult vähe pilvine ja soojemal hooajal, kui pole lund ja puud on lehteis. Teiseks on käesolev pilt heaks aluseks, sest sellel on sattunud raie- ja metsapiirkondi kõrvuti, mis võimaldab erinevatesse klassidesse kuuluvate alade üleminekukohti analüüsida.

\begin{figure}[H]
    \centering
    \includegraphics[width=.7\textwidth]{figures/seose_leidmine/näidisSadeliidiPilt.png}
    \caption{Näidis sateliidi pilt}
    \label{fig:näidisSadeliidiPilt}
\end{figure}

Joonisel \ref{fig:raieInfoMask} on näha lageraie piirkond, mis on toodud välja tumedama roosana, ja metsa piirkond, mis on joonisel heleroosana. Tegemist on aladega, mis on teada Keskkonnaametile ja mille kohta on olemas ka metsateatis. Satelliidipilt on saadud Sentinel-2 andmebaasist ja sellel on kümnemeetrine resolutsioon. Antud pilt on saadud raie teostamise ajast kuni 40 päeva hiljem, seega mahub õigesse ajaraamistikku, et piisavalt täpselt hinnata metsade seisukorda.

\begin{figure}[H]
    \centering
    \includegraphics[width=.7\textwidth]{figures/seose_leidmine/raieInfoMask.png}
    \caption{Raie piirkonna mask sateliidi pildil}
    \label{fig:raieInfoMask}
\end{figure}

Joonisel \ref{fig:raieInfoMask_ekspert} on näha lageraie piirkond, mis on toodud välja tumepunasena ja metsa piirkond, mis on märgitud rohelisega. Ekspert on võtnud metsateatistest saadud info alade tuvastamisel aluseks, aga sellele lisaks kasutas ta ka muud teavet nagu ortofotosid ja erinevate lainepikkuste pilte, et leida täpsemad piirded metsade ja raiealade vahel. Ekspert on leidnud, et antud piirkonnas leidub alasid, mis ei vasta metsateatistes märgendatud infole. Näiteks on pildil alasid, mis teatistes on märgitud metsaks, aga silmaga vaadates kujutab hoopis raiet ja ka vastupidi. Seega on eksperdi poolt loodud mask palju täpsem ja seetõttu oli eksperdi kaasamine vajalik andmestiku loomise protsessis.

\begin{figure}[H]
    \centering
    \includegraphics[width=.7\textwidth]{figures/seose_leidmine/raieInfoMask_ekspert.png}
    \caption{Raie piirkonna mask eksperdi poolt korrigeeritud}
    \label{fig:raieInfoMask_ekspert}
\end{figure}

Järgneval joonisel \ref{fig:segmenteeritudPealiskiht}, rakendati klastrianalüüsi süvaõppemudelist saadud
kõrgdimensionaalsetele tunnusevektoritele, mis esindavad sisendpildi
diskreetseid paiku. Eelkõige kasutati k-keskmise algoritmi, et grupeerida need
tunnusevektorid sarnaste semantiliste omaduste alusel. Igale pildilaigule vastav
tunnusevektor määrati ühte eelnevalt defineeritud arvu klastritesse, mille
tulemusena saadi diskreetne klastermärgis iga pildipaiga jaoks. Seejärel
visualiseeriti saadud klastermärgis pildil värvilise maskina, kus iga klaster on
esitatud unikaalse värviga, võimaldades seeläbi kvalitatiivselt hinnata mudeli
õpitud representatsioonide kohalikku sarnasust sisendpildil.

\begin{figure}[H]
    \centering
    \includegraphics[width=.7\textwidth]{figures/seose_leidmine/segmenteeritudNäidis.png}
    \caption{Segmenteeritud näidis sateliidi pildist}
    \label{fig:segmenteeritudPealiskiht}
\end{figure}

Joonisel \ref{fig:tsneDinoPatchEmbedings} on rakendatud klassipõhist klasterdamise meetodit, mille eesmärk on luua
pildimaterjalist semantiliselt sidusaid piirkondi, grupeerides pildi elemendid
eelnevalt defineeritud klassifikatsioonikategooriate alusel. Lähtudes mudeli
genereeritud klassifikatsiooniväljunditest, määratakse iga pildielement kõige
tõenäolisemasse klassi, mille alusel moodustatakse klastrid. Selle meetodi
rakendamise tulemusena saadakse segmentatsioon, kus ühte klastrisse kuuluvad
pildi osad on mudeli poolt klassifitseeritud sarnaselt. Lõppeesmärk on seeläbi
genereerida pildist arusaadavam representatsioon, mis
võimaldab interpreteerida pildi sisu semantilisel tasemel, tuues
esile objektide ja piirkondade klassipõhised seosed. Pildil on väljatoodud klass 1 mis on mets ja klass 2 mis on lageraie, lisaks sai eemaldatud tausta klass, et oleks kergem jälgida uuritavaid klasse.

\begin{figure}[H]
    \centering
    \includegraphics[width=.7\textwidth]{figures/seose_leidmine/tsneDinoPatchEmbedings.png}
    \caption{T-SNE kluster analüüs DinoV2 mudeli väljunditest}
    \caption*{kollane - mets, lilla - lageraie, roheline - taust}
    \label{fig:tsneDinoPatchEmbedings}
\end{figure}

Siit võis näha, et lageraie ja mets on eristatavad ja annavad alust edasi uurida, millised alusmudelid suudavad paremini eristada lageraie ja metsa alasid.

\section{Treenimis protsetuurid}

\chapter{Tulemuste analüüs}\label{chapter:analüüs}
Selles peatükis analüüsime eelmises peatükis kirjeldatud katsete tulemusi, et mõista, kuidas erinevad mudelikonfiguratsioonid ja treeningstrateegiad mõjutavad metsastunud ja lageraielõikude tuvastamise täpsust satelliitpiltidelt. Arutatakse ka tulemuste statistilist olulisust ning võrreldakse erinevate lähenemiste efektiivsust.
\section{Tulemuste võrdlus}
\textbf{Baasjoon}

Eksperimendi käigus testiti süstemaatiliselt erinevaid hüperparameetreid, et leida optimaalne konfiguratsioon antud ülesande jaoks. 
Esitatud tulemuste analüüs näitas, et parimad mudelid
saavutasid märkimisväärselt kõrgeid Dice'i skoore, ulatudes kuni \textasciitilde
0.984. See on eriti üllatav, arvestades andmestiku koostamise väikest valimit.

Parimaks osutunud mudeli konfiguratsioon oli järgmine:
\begin{itemize}
  \item \textbf{Arhitektuur:} Unet
  \item \textbf{Enkooder:} ResNet50
  \item \textbf{Enkooderi kaalud:} ImageNet (eelkoolitatud)
  \item \textbf{Optimeerija:} AdamW
  \item \textbf{Õpisamm:} \textasciitilde 1.09e-4
\end{itemize}

Ka teised kombinatsioonid,
näiteks DeepLabV3 koos ResNet50-ga, saavutasid kõrgeid tulemusi (Dice > 0.97).
Osaliselt on tipptulemused visualiseeritud ka lisatud joonisel \ref{fig:segmentation_results}.

\textbf{Kõrgete skooride põhjused}
Kõrgeid tulemusi väikesel andmestikul võib seletada mitme teguriga.
Siirdõpe (\textit{Transfer Learning}): ImageNet andmestikul eelkoolitatud enkooderite
kasutamine on tõenäoliselt peamine edu võti. Eelnevalt treenitud mudelid omavad
juba võimekust tuvastada üldiseid visuaalseid mustreid (nt servad, tekstuurid),
mida saab efektiivselt kohandada spetsiifilisele metsanduse segmenteerimise
ülesandele. See vähendab oluliselt vajamineva treeningandmestiku mahtu. Peaaegu
kõik parimad tulemused saavutati just imagenet kaaludega. 
Kuigi valideerimistulemused on kõrged, on väikse andmestiku kasutamine riskantne, sest alati peab arvestama ülesobitamise (\textit{overfitting}) ohuga. Samas näitab analüüs, et paljudel juhtudel valideerimis- ja treeningkahju
(val\_loss, train\_loss) vähenesid sünkroonis, mis viitab sellele, et mudelid
suutsid siiski valideerimisandmetele edukalt üldistuda ega õppinud
treeningandmeid lihtsalt pähe.

\begin{figure}[H]
    \centering
    \includegraphics[width=0.8\textwidth]{figures/top3_dice.png}
    \caption{Parimad DICE tulemused erinevate mudelite vahel.}
    \label{fig:segmentation_results}
\end{figure}

Samas individuaalsed klassi tulemused (nt metsastumine vs lageraie) Dice skoorid näitasid suuresti, et mudel ei suuda
täpselt eristada metsaga kaetud alasid ja lageraielõike. Kõrge skoor tuleneb suuresti tagatausta suurest osakaalust pildil. Joonisel \ref{fig:sidebyside_forest_bg} on näha et ülejäänud tausta keskmine Dice skoor on 0.98, mis on väga kõrge. Aga klassipõhised tulemused näitavad, et metsade ja lageraie alade eristamine on keeruline. Mudelite katsed ka näitavada, et ka parimate mudelite puhul ei suuda nad metsa ja lageraiet eristada.

\begin{figure}[H] % Placement specifier: h-here, t-top, b-bottom
    \centering
    \begin{subfigure}[b]{0.8\textwidth} % Set width for side-by-side arrangement
        \includegraphics[width=\textwidth]{figures/tulemused/dice_per_class_y_forest.png} % Image path and full width in subfigure
        \caption{} % Leave caption empty to automatically label this subfigure as (a)
        \label{fig:dice_per_class_y_forest}
    \end{subfigure}
    % Second subfigure
    \begin{subfigure}[b]{0.8\textwidth} % Set width for side-by-side arrangement
        \includegraphics[width=\textwidth]{figures/tulemused/dice_per_class_background.png} % Path and full width in subfigure
        \caption{} % Leave caption empty to automatically label this subfigure as (b)
        \label{fig:dice_per_class_background}
    \end{subfigure}
    
    \caption{Tagatausta ja noore metsa segmentatsiooni Dice tulemused üle katsete} % Main caption for the whole figure environment
    \label{fig:sidebyside_forest_bg} 
\end{figure}




\textbf{Dinov2 eksperimendid}

Käesolevas uurimuses viidi läbi eksperimente, mille eesmärk oli hinnata DINOv2
raamistikul põhinevate masinõppemudelite efektiivsust satelliidipiltide
segmenteerimisel metsanduslikus kontekstis. Katsetati erinevaid Vision
Transformer (ViT) arhitektuure (nt ViT-B/14, ViT-G/14, ViT-L/14, ViT-S/14) koos
erinevate segmenteerimispeadega (nt LinearHead, SimpleHead, FPNHead,
UPerNetHead). Mudelite jõudlust hinnati keskmise Dice'i koefitsiendi, keskmise
IoU (Intersection over Union) ja keskmise täpsuse (Mean Accuracy) alusel,
jälgides neid mõõdikuid kuni 800 epohhi vältel. Treeningutingimustes varieeriti
hüperparameetreid, nagu õppimismäär (nt 1e-5, 5e-5, 1e-4), partii suurus (1, 2,
4) ning rakendati nii külmutamist (Freeze) kui ka peenhäälestamist (FineTune).

\bigskip
\begin{longtable}{llll}
    \textbf{Konfiguratsioon} & \textbf{Mean Dice} & \textbf{Mean IoU} & \textbf{Mean Täpsus} \\
    \hline
    ViTB14 LinearHead LR5e-5 BS2 E5 Freeze & 0.18 & 0.16 & 0.26 \\
    ViTB14 SimpleHead LR1e-5 BS4 E5 FineTune & 0.200 & 0.26 & 0.35 \\
    ViTG14 FPNHead LR1e-5 BS4 E5 FineTune & 0.258 & 0.28 & 0.33 \\
    ViTG14 SimpleHead LR1e-4 BS4 E5 Freeze & 0.22 & 0.275 & 0.350 \\
    ViTL14 UPerNetHead LR1e-5 BS1 E3 Freeze & 0.160 & 0.270 & 0.34 \\
    ViTS14 FPNHead LR1e-5 BS4 E5 FineTune & 0.23 & 0.25 & 0.38 \\
    \hline
    \caption{DINOv2 eksperimendi tulemused}
    \label{tab:dinov2_results}
\end{longtable}
\bigskip

Eksperimentide tulemused näitavad, et DINOv2-põhiste mudelite jõudlus satelliidipiltide segmenteerimisel sõltub oluliselt närvivõrgu selgroo suurusest, segmenteerimispea keerukusest ja treeningustrateegiast. Suuremad närvivõrkude selgrood (nt ViTG14) ja mitme skaala tunnuseid töötlevad pead (nt FPNHead) andsid parimaid tulemusi, saavutades keskmise IoU kuni 0,28 ja täpsuse kuni 0,38. Peenhäälestamine parandas jõudlust võrreldes külmutatud mudelitega, kuid pikaajalistes treeningutes (500–800 epohhi) ilmnes ebastabiilsus, mis viitab ületreenimisele või ebapiisavale regulatsioonile.

Soovitused edasiseks uurimistööks hõlmavad varajase peatamise ja õppimismäära ajakava rakendamist stabiilsuse suurendamiseks, samuti täiendavate arhitektuuride (nt hübriidmudelite) testimist jõudluse lae ületamiseks. Praegused tulemused pakuvad väärtuslikku alust metsanduslike satelliidipiltide segmenteerimise optimeerimiseks.

\textbf{SAMv2 eksperimendid}

Sarnaselt DINOv2 katsetele viidi läbi ka eksperimendid Segment Anything Model (SAMv2) raamistiku kasutamisel. SAMv2 on tuntud oma võime poolest tegeleda erinevate segmentatsiooniülesannetega, sealhulgas satelliidipiltide analüüsiga. Katsetes uuriti erinevaid hüperparameetreid, sealhulgas õppimismäära, partii suurust ja treeninguetappe, et leida parim konfiguratsioon metsanduse kontekstis.

Tabelis on seitse eksperimenti, mis erinevad mudeli variandi, epohhide arvu,
õppimiskiiruse ja segmenteerimispea tüübi poolest. Kõik eksperimendid kasutavad
501 epohhi. Parima konfiguratsiooni leidmiseks võrdleme IoU ja Dice väärtusi, kuna need on peamised näitajad mudeli jõudluse hindamisel. Tabelist selgub, et kõrgeimad tulemused on järgmised:

\begin{itemize}
    \item Kõrgeim IoU:  0.350205 (konfiguratsioon: large, õppimiskiirus $1 \times 10^{-5}$, Multi-Layer pea).
    \item Kõrgeim Dice:  0.262514 (sama konfiguratsioon: large, õppimiskiirus $1 \times 10^{-5}$, Multi-Layer pea).
\end{itemize}

\begin{longtable}{ccccccc}
    \hline
    Variant & Epochs & Learning Rate & Head Type & IoU & Dice \\
    \hline
        tiny &  501 & $1 \times 10^{-4}$ & Simple & 0.229138 & 0.150820\\
        tiny &  501 & $1 \times 10^{-4}$ & Multi-Layer & 0.256366 & 0.178527 \\
        large &  501 & $1 \times 10^{-4}$ & ASPP & 0.327914 & 0.246620  \\
        large &  501 & $1 \times 10^{-4}$ & U-Net Style & 0.289425 & 0.205382  \\
        large &  501 & $1 \times 10^{-5}$ & Multi-Layer & 0.350205 & 0.262514 \\
        base\_plus & 501 & $1 \times 10^{-4}$ & Multi-Layer & 0.279433 & 0.195139  \\
        large &  501 & $1 \times 10^{-4}$ & Simple & 0.242327 & 0.163633 \\
    \hline
\caption{Samv2 eksperimendi tulemused}
\label{tab:samv2_results}
\end{longtable}

Järgnevas joonises on välja toodud SAMv2 eksperimendi tulemused eksperimendi lõikes ja jaotadud klassideks.

\begin{figure}[H]
    \centering
    \includegraphics[width=0.8\textwidth]{figures/tulemused/Sam_experimentide_tulemused_klassid.png}
    \caption{SAMv2 eksperimendi tulemused klasside lõikes. }
    \label{fig:sam_result}
\end{figure}

Kasutades eelmises osas treenitud SAMv2 mudelit, genereerisime segmentatsioonikaardid:
\begin{figure}[H]
    \centering
    \includegraphics[width=0.8\textwidth]{figures/tulemused/prediction_3403037_TCI.png}
    \caption{SAMv2 pareima mudeli segmentatsioonitulemused}
    \label{fig:sam_best_result}
\end{figure}

Metsateatistel põhinev segmentatsiooni tulemus:
\begin{figure}[H]
    \centering
    \includegraphics[width=0.6\textwidth]{figures/tulemused/ground_truth_mask_3403037.png}
    \caption{Metsateatistel põhinev segmentatsiooni tulemus}
    \label{fig:actual_mask}
\end{figure}



\textbf{Analüüsi kokkuvõte}

Kokkuvõttes näitavad eksperimendid, et DINOv2 ja SAMv2 raamistike kasutamine satelliidipiltide segmenteerimisel metsanduse kontekstis võib anda häid tulemusi, kuid nõuab hoolikat hüperparameetrite valikut ja treeningstrateegiate rakendamist. Parimad tulemused saavutati suuremate närvivõrkude selgroogude ja keerukamate segmenteerimispeadega, mis suudavad efektiivselt töödelda erinevaid visuaalseid mustreid. Tulevikus võiks uurida ka hübriidmudelite kasutamist, mis kombineerivad erinevate arhitektuuride eeliseid.

\section{Edasiarendus ja täiustamine}
Ehkki käesolev töö demonstreeris isejuhitud Vision Transformer'i põhisele DINO
v2 mudelile ja SAM v2 väheste õppeandmete lähenemisele tugineva segmentatsiooni
edukust Sentinel-2 piltidel, on olemas mitmeid võimalusi täiendusteks ja
metoodilisteks parendusteks:

\textbf{Andmestiku suurendamine}
Käesolevas uurimuses kasutati 15 lageraie polügooni. Tulevastes töödes võiks
andmestikku oluliselt laiendada, kaasates rohkem lageraie juhtumeid ja
erinevaid metsatüüpe. Samuti käesolevas töös ei võetud arvesse hooajalisi
muutusi, seega võiks lisada andmeid erinevatest aastaaegadest ja ilmastikutingimustest,
et suurendada mudeli üldistusvõimet. Näiteks võiks isegi luua mitu mudelit, eri aastaaegade ja ilmastikutingimuste jaoks, et saavutada paremad tulemused.

\textbf{Super-resolutsiooni rakendamine} 
Kuna Sentinel-2 ruumiline lahutusvõime on
piiratud (10 m/piks), võib väikeste lageraie fragmentide täpsemaks tuvastamiseks
kasutada süvaõppel põhinevaid super-resolutsiooni mudeleid (nt ESRGAN, RCAN).
Eeltehtud super-resolutsiooni mudelid suudaksid tõsta sisendi detailsust,
võimaldades segmenteerimisel paremini eristada kitsaid raiemustreid ja segatud
taimestiku.

\textbf{Rohkemate spektraalribade kaasamine} Uuringus keskendusime 10 m ja 20 m
riba­komplektile (B02--B08), kuid Sentinel-2 andmestikul on ka 60 m
lahutusvõimega B01, B09, B10 ribad. Nende ja teiste indeksite (nt NDMI, EVI,
SAVI) kombineerimine võib lisada eristavust pilvede, märgade alade ja erinevate
puuliikide vahel. Lisaks tasub uurida Sentinel-1 S-banda radariandmete liitmist,
et parandada pilvekatte all segmentatsiooni ja tuvastada aluspinnamuutusi.

\textbf{Kõrgema resolutsiooniga satelliidiandmete integreerimine} Kommertssatelliidid
nagu WorldView-3 (0,3 m piks), Pleiades (0,5 m) ja PlanetScope (3 m) pakuvad
oluliselt paremat ruumilist lahutust. Nende kõrglahutusega andmete
mitmespektriline analüüs (nt PAN-sharpening) võib toetada treeningandmete
loomist ja mudeli peenhäälestust, võimaldades tuvastada ja valideerida
väiksemaid raietegevuse detaile.

\textbf{Ajarealist analüüsi süvendamine} Käesolev töö käsitles iga lageraie juhtumit
eraldi hetke põhiselt. Järgmistes etappides võiks integreerida
ajaseeria-segmendatsiooni (nt ConvLSTM, Temporal CNN), et modelleerida metsa
arengut ja raiemustreid ajas. See võimaldaks tuvastada hooajalisi ja
aastate-üleseid trende ning eristada ajutisi muutusi (nt varisemine, põua- või
tulekahju-kahjustused) tegelikest lageraietest.

\textbf{Täiustatud pilve- ja varjumaskimine} Pilve ning varjude eemaldamine piiras osa
piltide kasutamist. Tulevases töös võiks katsetada nn ``cloud shadow''
tuvastamise meetodeid (nt Fmask, Sen2Cor täiendused) või sügavõppel põhinevaid
pilvemaskimudeleid, et taastada katkenud ajaread ja saavutada pidevam andmevoog.

\textbf{Mudeli peenhäälestuse automatiseerimine} Hüperparameetrite käsitsi otsing on
ajamahukas. Võiks rakendada Bayesia optimeerimist või evolutsioonialgoritme (nt
Optuna, Hyperopt) hüpermäärade leidmiseks, samuti kasutada automaatset Data
Augmentation strateegiate valikut (nt AutoAugment), et tugevdada mudeli
üldistusvõimet väheste andmetega.

\textbf{Üldistusvõime ja siirdõpe uutes piirkondades} Järgnevalt on soovitatav hinnata
mudeli ülekantavust erinevates geograafilistes ja metsatüüpides (nt
okaspuumetsad Põhja-Soomes, troopilised metsad Indoneesias). Siirdõpe (transfer
learning) sobiva allmudeliga ja peenhäälestus väheste lokaalsete näidete põhjal
aitaks välja selgitada, millises mahus mudel vajab uut treeningandmestikku
globaalseks rakenduseks.


\chapter{Kokkuvõte}\label{chapter:kokkuvõte} 
Käesolevas magistritöös uuriti isejuhitud Vision Transformer'ite ja väheste
õppeandmete lähenemiste sobivust lageraie tuvastamiseks Sentinel-2
multispektraalsetelt satelliidipiltidelt. Esiteks loodi ainulaadne andmestik,
mis koosneb 15 lageraie polügoonist ja nende ümbrusest, kombineerides riiklikud
metsateatised täiendava eksperthinnangu ning K-keskmise klasterdamise alusel
diferentseeritud okas- ja lehtpuu alad. Järgnenud eeltöötlemise sammudes laaditi
alla sobivad Sentinel-2 Level-2A tooted, teostati atmosfääriline korrektsioon
ning genereeriti georeferentseeritud GeoTIFF rasterfragmendid.

Seejärel peenhäälestati ja võrreldi kolme tipptasemel segmentatsiooniraamistiku
(DINO v2 + SAM v2, U-Net, Segmentation Model Zoo modulaarpead: DeepLabv3, FPN)
tulemuslikkust, mõõdetuna peamiselt IoU ja Dice koefitsiendiga. Tulemused
näitasid, et DINO v2 baasil ühendatud väheste näidete treeningu skeem
pakkus kõige paremat segmenteerimistäpsust. U-Net jäi veidi alla, kuid
kergesti peenhäälestatavate hüperparameetrite tõttu osutus produktiivseks
lähtepunktiks.

Uurimistöö tõestas, et isejuhitud peaahelad suudavad väheste andmetega anda
küll kesiseid tulemeid satelliidipildisegmenteerimise valdkonnas,
vähendades sõltuvust suurest märgendatud andmestikust ja eksperthõlmapanust.
Samas toodi välja mitmed potentsiaalsed suunad edasiseks arenduseks, sh
super-resolutsiooni rakendamine, täiendavate spektraalribade ja
kommerts-satelliitide andmete kaasamine, ajarealise analüüsi süvendamine ning
pilve- ja varjumaskimise strateegiate täiustamine.

Kokkuvõttes annab töö aluse automaatsete algoritmide kasutusele metsaraiete
täpsemaks jälgimiseks ja metsade jätkusuutlikuks haldamiseks. Edasised uurimused
võiksid laiendada metoodikat globaalsesse konteksti ja integreerida
mitmeallikalisi andmeid, et saavutada veelgi detailsem ja vastupidavam
metsanduse seire- ning valduse haldamise lahendus.

\zlabel{lastpagetocount}        % DO NOT REMOVE! Used for counting number of pages of main text
