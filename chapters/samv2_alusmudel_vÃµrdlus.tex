Segment Anything Model 2 (SAM 2) on pildi- ja videosegmentatsiooni ülesannete
fundamentaalne mudel, mis laiendab esmalt välja töötatud SAM-i  (Kirillov et al., 2023 \cite{kirillovSegmentAnything2023})
 võimekust staatilistel piltidel ka ajaliselt dünaamilisele
videomaterjalile. SAM 2 arhitektuur põhineb lihtsal transformeril, millel on
integreeritud vooguv mälumoodul --- see võimaldab reaalajas töödelda järjestikku
sisenevaid kaadriseid ja säilitada informatsiooni sihtrühma objektist.
Mälumoodul hoiab kuni N varasemat kaadrit ning M kasutajapõhist sisendkaadrit,
mille alusel ennustatakse masklete'e (masklet = objekti mask spatio-temporaalses
ulatuses). \cite{raviSAM2Segment2024}

SAM v2 põhistruktuur koosneb neljast põhimoodulist, mis töötavad järjestikku, et
ühendada nii pildist tulenev informatsioon kui ka kasutajapoolne juhis ning
ajaliselt varasemad tulemused. Esiteks võtab hierarhiline pildikodeerija sisse
sisendkaadri ja teisendab selle mitme skaalaga
omadusvektoriteks. Teiseks kodeerib prompt-moodul kasutaja märgitud punkte,
kaste või maski, et mudel teaks, mida täpselt otsida. Kolmandaks lisab
mälumoodul iga kaadri puhul salvestatud maski- ja pildiomadused FIFO-põhimõttel
mälupanka, võimaldades modellel säilitada konteksti eelmistest kaadritest.
Neljandaks rakendab transformer-põhine tähelepanumehhanism nii jooksva kaadri
kui ka mälust saadud featuuride vahelise suhtluse, mille tulemuseks antakse
maskidekooderile täpsed ruumilised vihjed. Lõpuks dekodeerib maskidekooder
paralelselt kõik saadud sisendid ning ennustab lõpliku objektimaski, mida mudel
iteratiivselt uuendab ja talletab järgmiste kaadrite jaoks. Selline ülesehitus
võimaldab SAM v2-l kiiresti ja tõhusalt segmenteerida nii staatilisi pilte kui
ka videoid, säilitades minimaalse kasutajainteraktsiooni juures detailse täpsuse.
SAM 2 rakendused katavad laia valikut domeene: meditsiiniline pilditöötlus,
kaugseire ja satelliitpildi analüüs, liikumisskeeme segmenteerivad uuringud ning
maskeeritud objektide tuvastamine. \cite{raviSAM2Segment2024}

Artiklis  \glqq MPG-SAM 2: Adapting SAM 2 with Mask Priors and Global Context for Referring Video Object Segmentation\grqq{} kohandati SAM V2 mudelit viitematerjalidel põhinevaks
videoobjektide segmentimiseks (RVOS). Teadlased lõid multimodaalse enkooderi,
mis ühendas video- ja tekstipõhiseid tunnuseid, ning kasutasid maskide
eelgeneraatoreid globaalse konteksti kaardistamiseks. Uuenduslik hierarhiline
ajalooline agregeerimismoodul võimaldas SAM 2-l säilitada objekti kontuure läbi
videokaadrite, parandades ajaliselt järjepidevat segmentatsiooni. Katsetused
mitmetel RVOS andmestikel näitasid märgatavat täpsuse kasvu võrreldes varasemate
meetoditega. \cite{rongMPGSAM2Adapting2025}

Artiklis \glqq SAM2 for abdomen: One-shot and no finetuning\grqq{} esitletakse SAM2 ja Emb-SAM
 meetodite kombinatsiooni, mis võimaldab
kõhuelundite segmentatsiooni CT (\textit{computed tomography}) kujutistel, kasutades ainult ühte märgistatud
pilti. SAM2 käsitleb järjestikuseid CT-viilusid kui videosarju ning
propagatsioon toimub maskimälu kaudu. Emb-SAM täiendab protsessi, luues täpseid
pseudo-märkeid iseenesliku õppimisega. Tulemused BTCV andmestikul näitavad
tugevat segmentatsioonitäpsust, minimeerides käsitsi sekkumise vajadust. \cite{hwangSAM2AbdomenOneshot2024}