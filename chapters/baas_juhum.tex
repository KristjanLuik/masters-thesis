Selles lõigus tutvustame lühidalt kasutatud aluspõhimõtteid ja arhitektuure:
ResNet34, ResNet50 ning MobileNetV2 kodeerijad, optimeerijad Adam ja AdamW ning
segmentatsioonipead U-Net, DeepLabv3 ja FPN\@. ResNet-põhised arhitektuurid
rakendavad jääkühikuid, mis võimaldavad väga sügavate võrkude stabiilset
treenimist. MobileNetV2 kasutab pööratud jääkühikuid ja lineaarseid pudelikaelu,
et tagada mobiilsete seadmete jaoks kõrge efektiivsus . Adam-optimeerija pakub
adaptatiivseid esimest ja teist momenti põhinevaid õppemäärade uuendusi , samas
kui AdamW eraldab kaalude hääbumise gradientiuuendusest . U-Net koosneb
kokkutõmbuvast ja laienevast teest , DeepLabv3 kasutab atrooskonvolutsioone ning
astrusruumilist pirnikogumit (ASPP), ning FPN loob mitmetasandilise omaduskaardi
püramiidi, mis sobitub hästi kõrge lahutusvõimega satelliitpiltidega. Väärtuste
kombineerimine ja ulatuslik hüperparameetrite otsing erinevate kodeerijate,
optimeerijate ja segmentatsioonipeade vahel annab võimaluse modelleerimisel
täpsust ja efektiivsust optimeerida.

\textbf{Kodeerijad (Encoders)}

\textbf{ResNet34}

ResNet34 on 34-kihiline sügav närvivõrk, mis põhineb VGG-19 inspireeritud „plain” arhitektuuril, kuhu on lisatud jääkühikud (residual blocks) ([GeeksforGeeks][1]). Iga jääkühik sisaldab kahe või enama konvolutsioonikihti, mille väljund liidetakse sisendiga „lünkühenduse” (skip connection) kaudu, mis aitab vältida gradientide kadu väga sügavate võrkude puhul ([viso.ai][8]). Spetsiifiliselt ResNet34 puhul on kasutusel järjestus 3×3 konvolutsioone ning iga lünkühik korrigeerib sisendi ja väljundi vahelist erinevust, võimaldades algset teavet säilitada.


\begin{figure}[H]
    \centering
    % === ResNet34 Residual Block ===
    \begin{tikzpicture}[node distance=8mm, auto, >=Stealth]
    % Nodes
    \node (input)   [draw, rectangle]                  {Sisendomadused};
    \node (conv1)   [draw, rectangle, below=of input]  {3×3 Conv + BN + ReLU};
    \node (conv2)   [draw, rectangle, below=of conv1]  {3×3 Conv + BN};
    \node (add)     [draw, circle, right=of conv2, xshift=16mm] {+};
    \node (relu)    [draw, rectangle, below=of add]    {ReLU};
    \node (output)  [draw, rectangle, below=of relu]   {Väljund};
    
    % Arrows
    \draw[->] (input) -- (conv1);
    \draw[->] (conv1) -- (conv2);
    \draw[->] (conv2) -- (add.west);
    \draw[->] (input.east) to[out=0, in=90] (add.north);
    \draw[->] (add) -- (relu);
    \draw[->] (relu) -- (output);
    \end{tikzpicture}

    \caption{ResNet34 jääkühikust skeem}
    \label{fig:ResNet34jääkühikust}
\end{figure}

\textbf{ResNet50}

ResNet50 laiendab jääkühikute ideed kolmik-kihiliste pudelikaelablokkidega: 1×1 konvolutsioon vähendab dimensiooni, 3×3 konvolutsioon töötleb omadusi ja 1×1 konvolutsioon taastab dimensiooni ([Medium][9]). See arhitektuur võimaldab sügavust tõsta 50 kihini, hoides samal ajal arvutuskoormust kontrolli all ([viso.ai][8]). Põhimõte on sama – lünkühendus ühendab sisendi väljundiga, et säilitada gradientide voo kogu võrgu ulatuses.

% === 1. U-Net segmentatsioonipea ===
\begin{figure}[!ht]
    \centering
    \begin{tikzpicture}[node distance=8mm, auto, >=stealth]
    % Nodes
    \node (input) [draw, rectangle] {Sisendomadused};
    \node (conv1) [draw, rectangle, below=of input] {1×1 Conv};
    \node (conv3) [draw, rectangle, below=of conv1] {3×3 Conv + ReLU × n};
    \node (conv1b) [draw, rectangle, below=of conv3] {1×1 Conv};
    \node (upsample) [draw, rectangle, below=of conv1b] {Upsample ×2};
    \node (concat) [draw, rectangle, below=of upsample] {Concat (encoder-layer)};
    \node (finalconv) [draw, rectangle, below=of concat] {3×3 Conv + ReLU};
    \node (output) [draw, rectangle, below=of finalconv] {Ennustatud mask};

    % Arrows
    \draw[->] (input) -- (conv1);
    \draw[->] (conv1) -- (conv3);
    \draw[->] (conv3) -- (conv1b);
    \draw[->] (conv1b) -- (upsample);
    \draw[->] (upsample) -- (concat);
    \draw[->] (concat) -- (finalconv);
    \draw[->] (finalconv) -- (output);
    \end{tikzpicture}
    \caption{ResNet50 jääkühikust skeem}
    \label{fig:ResNet50jääkühikust}
\end{figure}

\textbf{MobileNetV2}

MobileNetV2 on kavandatud eelkõige mobiilseadmete jaoks, kasutades pööratud jääkühikuid ja lineaarseid pudelikaelu ([arxiv.org][2]). Pööratud jääkühikud algavad kitsast sisendkanalist, laiendavad omaduste ruumi läbi 1×1 konvolutsiooni, töötlevad seda 3×3 süvendkonvolutsiooniga (depthwise convolution) ja vähendavad seejärel lineaarse pudelikaelaga dimensiooni ([openaccess.thecvf.com][10]). Lineaarne aktivatsioon pudelikaelal aitab vältida teabekadu kitsas kanaliruumis.

\begin{figure}[H]
    \centering
    % === MobileNetV2 Inverted Residual Block ===
    \begin{tikzpicture}[node distance=8mm, auto, >=Stealth]
    % Nodes
    \node (input)   [draw, rectangle]                    {Sisendomadused\,($d$ kanalit)};
    \node (exp1x1)  [draw, rectangle, below=of input]    {1×1 Conv, Expansioon $\times t$};
    \node (dw3x3)   [draw, rectangle, below=of exp1x1]    {3×3 Depthwise Conv + BN + ReLU6};
    \node (proj1x1) [draw, rectangle, below=of dw3x3]     {1×1 Conv, Projektsioon}; 
    \node (output)  [draw, rectangle, below=of proj1x1]   {Väljund\,($d'$ kanalit)};
    
    % Skip connection if strides=1
    \node (add)     [draw, circle, right=of dw3x3, xshift=16mm] {+};
    
    % Arrows
    \draw[->] (input)   -- (exp1x1);
    \draw[->] (exp1x1)  -- (dw3x3);
    \draw[->] (dw3x3)   -- (proj1x1);
    \draw[->] (proj1x1) -- (output);
    
    % Optional skip
    \draw[->] (input.east) to[out=0, in=90] (add.north);
    \draw[->] (dw3x3.east) to[out=0, in=180] (add.west);
    \draw[->] (add) -- (proj1x1);
    \end{tikzpicture}

    \caption{MobileNetV2 jääkühikust skeem}
    \label{fig:MobileNetV2jääkühikust}
\end{figure}

\textbf{Optimeerijad}

\textbf{Adam}

Adam (Adaptive Moment Estimation) on gradientipõhine optimeerija, mis kombineerib RMSProp-i ja momentum‘i eelised, kasutades kohandatud momentide hinnanguid ([arxiv.org][3]). Esimene moment (keskmine) ja teine moment (ruutkeskmine) hinnatakse eksponentsiaalse libiseva keskmisena, mis võimaldab õppemäära dünaamilist kohandamist igakaalu tasemel ([arxiv.org][11]). Adam on efektiivne suurte ja/või mürarohkete gradientide tingimustes ning vajab tavaliselt vähe hüperparameetrite häälestust.

\textbf{AdamW}

AdamW täiustab Adam-i, eraldades kaalude hääbumise (weight decay) gradientiuuendusest, mis parandab regulaarimist ja üldist jõudlust ([arxiv.org][4]). Kaalude hääbumine lisatakse otse update -sammule, mitte gradientide skaleerimise kaudu, tagades korrektse L2-regularisatsiooni teostuse ([paperswithcode.com][12]). See lahendab Adam-i algupärase L2 regulaarimise ühilduvusprobleemi.

\textbf{Segmentatsioonipead}

\textbf{U-Net}

U-Net koosneb sümmeetrilisest kokkutõmbuva ja laieneva radade (encoder-decoder) architektuurist, kus iga kokkutõmbuv tasand vähendab ruumilist lahutusvõimet ja suurendab tunnuste arv, ning vastupidises laienevas osas ruumiline lahutusvõime taastatakse ([arxiv.org][5]). Lünkühendused ühendavad kokkutõmbuva ja laieneva teekonna vastavad kihid, võimaldades detailse teabe säilimist ([SpringerLink][13]).

ASCII-skeem U-Neti blokist:

```
%[Sisend] → [Encoder] ───────────┐
%                             Concat → [Decoder] → [Väljund]
%[Sisend] → [Encoder] ────┘
```

\textbf{DeepLabv3}

DeepLabv3 kasutab atrooskonvolutsioone (dilated convolutions) erinevatel kihtidel, et laiendada vastuvõtupiirkonda ilma lahutusvõimet kahjustamata, ning Atrous Spatial Pyramid Pooling (ASPP) moodulit, mis kogub omadusi eri skaala tasanditel ([Medium][6]). See lähenemine võimaldab segmentatsioonil tuvastada objektide piirjooni täpsemalt, eriti keerukates satelliitpiltides ([paperswithcode.com][14]).

% === 2. DeepLabv3 segmentatsioonipea ===
\begin{figure}[!ht]
    \centering
    \begin{tikzpicture}[node distance=8mm, auto, >=stealth]
    % Nodes
    \node (input) [draw, rectangle] {Sisendomadused};
    \node (aspp) [draw, rectangle, below=of input, text width=4cm, align=center]
        {Atrous Spatial Pyramid Pooling\\(rate=6,12,18 + pool)};
    \node (concat) [draw, rectangle, below=of aspp] {Concat skaalad};
    \node (conv1) [draw, rectangle, below=of concat] {1×1 Conv};
    \node (upsample) [draw, rectangle, below=of conv1] {Upsample ×4};
    \node (conv3) [draw, rectangle, below=of upsample] {3×3 Conv + ReLU};
    \node (output) [draw, rectangle, below=of conv3] {Ennustatud mask};

    % Arrows
    \draw[->] (input) -- (aspp);
    \draw[->] (aspp) -- (concat);
    \draw[->] (concat) -- (conv1);
    \draw[->] (conv1) -- (upsample);
    \draw[->] (upsample) -- (conv3);
    \draw[->] (conv3) -- (output);
    \end{tikzpicture}
    \caption{DeepLabv3 ASPP moodul}
    \label{fig:DeepLabv3ASPP}
\end{figure}


\textbf{Feature Pyramid Network (FPN)}

FPN loob hierarhilise omaduskaardi püramiidi, kombineerides madalama lahutusvõimega sügavamate kihtide tugevaid semantilisi omadusi kõrgema lahutusvõimega pinnakihtide detailidega ([openaccess.thecvf.com][7]). See sobib eriti hästi kõrge lahutusvõimega satelliitpiltide segmentatsiooniks, kus on oluline korraga töödelda nii suuri objekte kui ka peeneid detaile ([mdpi.com][15]).

% === 3. FPN (Feature Pyramid Network) segmentatsioonipea ===

\begin{figure}[H]
    \centering
    \begin{tikzpicture}[node distance=26mm, auto, >=stealth]
    % Input feature maps
    \node (c5) [draw, rectangle] {C5};
    \node (c4) [draw, rectangle, left=of c5] {C4};
    \node (c3) [draw, rectangle, left=of c4] {C3};
    \node (c2) [draw, rectangle, left=of c3] {C2};

    % 1x1 Convs
    \node (p5_1x1) [draw, rectangle, below=of c5] {1×1 Conv};
    \node (p4_1x1) [draw, rectangle, below=of c4] {1×1 Conv};
    \node (p3_1x1) [draw, rectangle, below=of c3] {1×1 Conv};
    \node (p2_1x1) [draw, rectangle, below=of c2] {1×1 Conv};

    % P5 path
    \node (p5_add) [draw, rectangle, below=of p5_1x1] {Add + Upsample};
    \node (p5_conv) [draw, rectangle, below=of p5_add] {3×3 Conv};
    \node (p5) [draw, rectangle, below=of p5_conv] {P5};

    % P4 path
    \node (p4_add) [draw, rectangle, below=of p4_1x1] {Add + Upsample};
    \node (p4_conv) [draw, rectangle, below=of p4_add] {3×3 Conv};
    \node (p4) [draw, rectangle, below=of p4_conv] {P4};

    % P3 path
    \node (p3_add) [draw, rectangle, below=of p3_1x1] {Add + Upsample};
    \node (p3_conv) [draw, rectangle, below=of p3_add] {3×3 Conv};
    \node (p3) [draw, rectangle, below=of p3_conv] {P3};

    % P2 path
    \node (p2_conv) [draw, rectangle, below=of p2_1x1] {3×3 Conv};
    \node (p2) [draw, rectangle, below=of p2_conv] {P2};

    % Final concat & conv
    \node (concat) [draw, rectangle, below=of p3, yshift=-6mm, text width=3cm, align=center]
        {Concat (P2–P5)};
    \node (finalconv) [draw, rectangle, below=of concat] {3×3 Conv};
    \node (output) [draw, rectangle, below=of finalconv] {Ennustatud mask};

    % Connections
    \foreach \i/\j in {c5/p5_1x1, c4/p4_1x1, c3/p3_1x1, c2/p2_1x1}
        \draw[->] (\i) -- (\j);

    \draw[->] (p5_1x1) -- (p5_add);
    \draw[->] (p5_add) -- (p5_conv);
    \draw[->] (p5_conv) -- (p5);

    \draw[->] (p4_1x1) -- (p4_add);
    \draw[->] (p4_add) -- (p4_conv);
    \draw[->] (p4_conv) -- (p4);

    \draw[->] (p3_1x1) -- (p3_add);
    \draw[->] (p3_add) -- (p3_conv);
    \draw[->] (p3_conv) -- (p3);

    \draw[->] (p2_1x1) -- (p2_conv);
    \draw[->] (p2_conv) -- (p2);

    \draw[->] (p2) -- ++(0,-5mm) -| (concat);
    \draw[->] (p3) -- ++(0,-5mm) -| (concat);
    \draw[->] (p4) -- ++(0,-5mm) -| (concat);
    \draw[->] (p5) -- ++(0,-5mm) -| (concat);

    \draw[->] (concat) -- (finalconv);
    \draw[->] (finalconv) -- (output);
    \end{tikzpicture}
    \caption{FPN segmentatsioonipea}
    \label{fig:FPNsegmentatsioonipea}
\end{figure}


\textbf{Mudelite kombineerimine ja hüperparameetrite häälestus}

Mudelite kombinatsioon sõltub kolmest peamisest komponentist: kodeerijast, optimeerijast ja segmentatsioonipeast. Hüperparameetrite ruumiline uurimine hõlmab kõigi võimalike kombinatsioonide treenimist ning tulemuste võrdlust, mis tagab parima täpsuse ja üldistamisvõime. Hüperparameetrite otsing sisaldas järgmisi seadeid:

%* **Kodeerija**: `resnet34` | `resnet50` | `mobilenet_v2`
%* **Optimeerija**: `adam` | `adamw`
%* **Pea**: `unet` | `deeplabv3` | `fpn`
\begin{longtable}{ll}
    \hline
    \textbf{Hüperparameeter} & \textbf{Võimalikud väärtused} \\
    \hline
    encoder\_name      & \{resnet34, resnet50, mobilenet\_v2\} \\
    encoder\_weights   & \{imagenet, None\} \\
    lr                 & $[1\times10^{-5},\,1\times10^{-3}]$ \\
    batch\_size        & \{2\} \\
    optimizer          & \{adam, adamw\} \\
    head               & \{unet, deeplabv3, fpn\} \\
    \hline
    \caption{Hüperparameetrite kombinatsioonid}
    \label{tab:hyperparameter_kombinatsioonid}
\end{longtable}


Sellise 3 × 2 × 3 kombinatsiooni korral tuleb kokku 18 eraldi mudelit, mida hinnata näiteks IoU (Intersection over Union) ja täpsuse alusel. Treening seisneb esmajärjekorras kodeerija eel-treenitud kaalu lahtipakkimises (transfer learning), seejärel kogu võrgu peenhäälestuses, kasutades sobivat optimeerijat ning õppemäära skeemi (nt kosinustagasilangus).


\begin{figure}
    \centering
    \begin{tikzpicture}[node distance=10mm]
    % sõlmed
    \node (input)   [block]                     {Sisend\\(RGB pilt)};
    \node (encoder) [block, below=of input]     {Kodeerija\\(ResNet34/ResNet50/MobileNetV2)};
    \node (head)    [block, below=of encoder]   {Segmentatsiooni-pea\\(U-Net/DeepLabv3/FPN)};
    \node (output)  [block, below=of head]      {Maski ennustus\\(Mets vs. lageraie klassid)};

    % nooled
    \draw[arrow] (input)   -- (encoder);
    \draw[arrow] (encoder) -- (head);
    \draw[arrow] (head)    -- (output);
    \end{tikzpicture}
    \caption{Mudeli arhitektuuri skeem}
    \label{fig:model_architecture}
\end{figure}

\textbf{Järeldused}

Ülaltoodud arhitektuuride ja optimeerijate kombineerimine võimaldab luua segmentatsioonimudeleid, mis suudavad satelliitpilte tõhusalt töödelda ja tuvastada metsasid ning lageraie alasid. ResNet-baasilised kodeerijad pakuvad tugevat omadusloome võimekust, MobileNetV2 lisab efektiivsust piiratud ressursiga kasutuskeskkondades. Adam ja AdamW optimeerijad tagavad stabiilse converge’tumise, samas kui U-Net, DeepLabv3 ja FPN pakuvad erinevaid lähenemisi segmentatsioonipeade ülesehitamiseks. Laialdane hüperparameetrite grid search ning kombineeritud mudelite analüüs annab objektiivse aluse valida kõige sobivam konfiguratsioon metsade ja lageraielõikude tuvastamiseks.

