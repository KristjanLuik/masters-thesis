Ehkki käesolev töö demonstreeris isejuhitud Vision Transformer'i põhisele DINO
v2 mudelile ja SAM v2 väheste õppeandmete lähenemisele tugineva segmentatsiooni
edukust Sentinel-2 piltidel, on olemas mitmeid võimalusi täiendusteks ja
metoodilisteks parendusteks:

\textbf{Andmestiku suurendamine}
Käesolevas uurimuses kasutati 15 lageraie polügooni. Tulevastes töödes võiks
andmestikku oluliselt laiendada, kaasates rohkem lageraie juhtumeid ja
erinevaid metsatüüpe. Samuti käesolevas töös ei võetud arvesse hooajalisi
muutusi, seega võiks lisada andmeid erinevatest aastaaegadest ja ilmastikutingimustest,
et suurendada mudeli üldistusvõimet. Näiteks võiks isegi luua mitu mudelit, eri aastaaegade ja ilmastikutingimuste jaoks, et saavutada paremad tulemused.

\textbf{Super-resolutsiooni rakendamine} 
Kuna Sentinel-2 ruumiline lahutusvõime on
piiratud (10 m/piks), võib väikeste lageraie fragmentide täpsemaks tuvastamiseks
kasutada süvaõppel põhinevaid super-resolutsiooni mudeleid (nt ESRGAN, RCAN).
Eeltehtud super-resolutsiooni mudelid suudaksid tõsta sisendi detailsust,
võimaldades segmenteerimisel paremini eristada kitsaid raiemustreid ja segatud
taimestiku.

\textbf{Rohkemate spektraalribade kaasamine} Uuringus keskendusime 10 m ja 20 m
riba­komplektile (B02--B08), kuid Sentinel-2 andmestikul on ka 60 m
lahutusvõimega B01, B09, B10 ribad. Nende ja teiste indeksite (nt NDMI, EVI,
SAVI) kombineerimine võib lisada eristavust pilvede, märgade alade ja erinevate
puuliikide vahel. Lisaks tasub uurida Sentinel-1 S-banda radariandmete liitmist,
et parandada pilvekatte all segmentatsiooni ja tuvastada aluspinnamuutusi.

\textbf{Kõrgema resolutsiooniga satelliidiandmete integreerimine} Kommertssatelliidid
nagu WorldView-3 (0,3 m piks), Pleiades (0,5 m) ja PlanetScope (3 m) pakuvad
oluliselt paremat ruumilist lahutust. Nende kõrglahutusega andmete
mitmespektriline analüüs (nt PAN-sharpening) võib toetada treeningandmete
loomist ja mudeli peenhäälestust, võimaldades tuvastada ja valideerida
väiksemaid raietegevuse detaile.

\textbf{Ajarealist analüüsi süvendamine} Käesolev töö käsitles iga lageraie juhtumit
eraldi hetke põhiselt. Järgmistes etappides võiks integreerida
ajaseeria-segmendatsiooni (nt ConvLSTM, Temporal CNN), et modelleerida metsa
arengut ja raiemustreid ajas. See võimaldaks tuvastada hooajalisi ja
aastate-üleseid trende ning eristada ajutisi muutusi (nt varisemine, põua- või
tulekahju-kahjustused) tegelikest lageraietest.

\textbf{Täiustatud pilve- ja varjumaskimine} Pilve ning varjude eemaldamine piiras osa
piltide kasutamist. Tulevases töös võiks katsetada nn ``cloud shadow''
tuvastamise meetodeid (nt Fmask, Sen2Cor täiendused) või sügavõppel põhinevaid
pilvemaskimudeleid, et taastada katkenud ajaread ja saavutada pidevam andmevoog.

\textbf{Mudeli peenhäälestuse automatiseerimine} Hüperparameetrite käsitsi otsing on
ajamahukas. Võiks rakendada Bayesia optimeerimist või evolutsioonialgoritme (nt
Optuna, Hyperopt) hüpermäärade leidmiseks, samuti kasutada automaatset Data
Augmentation strateegiate valikut (nt AutoAugment), et tugevdada mudeli
üldistusvõimet väheste andmetega.

\textbf{Üldistusvõime ja siirdõpe uutes piirkondades} Järgnevalt on soovitatav hinnata
mudeli ülekantavust erinevates geograafilistes ja metsatüüpides (nt
okaspuumetsad Põhja-Soomes, troopilised metsad Indoneesias). Siirdõpe (transfer
learning) sobiva allmudeliga ja peenhäälestus väheste lokaalsete näidete põhjal
aitaks välja selgitada, millises mahus mudel vajab uut treeningandmestikku
globaalseks rakenduseks.