\section{Masinõppe meetodite kasutus kaugseires}
Käsitletava teemaga seotud, kuid teiste piirkondade põhjal loodud, uurimistööde analüüsimisel prooviti välja selgitada, milliseid meetodeid on üldiselt kaugseires kasutatud, sealjuures kas on ehitatud mudeleid nullist või kasutatud
valmis mudeleid. Samuti oli oluliseks eesmärgiks, välja selgitada, kas nendel juhtudel on kasutatud süvaõpet või mitte. Lisaks sooviti teada saada, mis satelliidi andmeid on varasemalt kasutatud ja kas eri lainepikkuste sidumine on andnud paremaid tulemusi.

Ukraina teadlaste poolt koostatud 2021. aastal välja antud artiklis \glqq Deep Learning for Regular Change
Detection in Ukrainian Forest Ecosystem With Sentinel-2\grqq{} kasutati Copernicus Sentinel-2 satelliidi pilte, mis sisaldasid kõrge resolutsiooniga (10 m) värvi- ja spektrikanaleid, sealhulgas NDVI ja NDMI indekseid, võimaldades jälgida metsamuutusi kuni 5-päevaste intervallidega. Andmekogum loodi käsitsi Kharkivi piirkonnas, kasutades mitut järjestikust pilti ja põhjalikku märgistust, et tagada täpne deforestsatsioonipiirkondade kaardistamine. Uurijad rakendasid süvaõppe meetodeid, kasutades mitut U-Neti varianti (näiteks UNet-diff, UNet-CH, UNet2D, UNet3D, Siamese U-Netid ja UNet-LSTM), et hinnata nii ajast sõltuvaid kui ka ühekordseid lähenemisviise. Eraldi rõhutati piltidevahelise erinevuse kasutamise eeliseid, mis parandas segmentatsioonitulemusi ning tõstis Dice ja F1 skoore. Lisaks ilmnesid uuringus olulised nüansid, nagu pilvekatte, hooajaliste muutuste ning geograafiliste lahknevuste mõju, mis nõudsid täiendavat andmete eeltöötlust. Huvitav on, et kuigi kõik mudelid näitasid potentsiaali, saavutavad UNet-diff ja UNet-CH kõige kõrgema täpsuse, pakkudes seeläbi tõenduspõhiseid lahendusi metsakatteta ala muutuste regulaarseks jälgimiseks. \cite{isaienkovDeepLearningRegular2021}

Uus-Meremaal läbi viidud ja 2024. aastal välja antud uurimistöös \glqq Developing a forest description from remote sensing: Insights from
New Zealand\grqq{} kasutati kõrglahutusega lennufotosid ning regionaalseid ALS-andmeid radiata männi metsade täpseks kaardistamiseks Uus-Meremaal. Analüüs tugines sügavõppepõhisel semantilise segmentatsiooni mudelil, mis kasutab DeepLabv3+ arhitektuuri koos ResNext-101 peaahelana (backbone), saavutades IoU väärtused 0,94, täpsuse 0,96 ja meeldetuletuse 0,98. Keeruliseks osutus aga noorte istikute tuvastamine, mille puhul "juvenile" klass (noored istutatud metsapiirkonnad) liideti "radiata" (küpsemad männi alad) klassiga. Lisaks sügavõppemudelile kasutati mitmemuutujalisi regressioonimudeleid metsade keskmise kõrguse, kogumahte ja vanuse hindamiseks, saavutades kõrged R2 väärtused. \cite{pearseDevelopingForestDescription2025}