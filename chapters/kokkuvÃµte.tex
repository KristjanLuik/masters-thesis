Käesolevas magistritöös uuriti isejuhitud Vision Transformer'ite ja vähestel
õppeandmetel põhinevate lähenemiste sobivust lageraie tuvastamiseks Sentinel-2
multispektraalsetelt satelliidipiltidelt. Esiteks loodi ainulaadne andmestik,
mis koosneb 15 lageraiet ja nende ümbrusest, kombineerides riiklikud
metsateatised täiendava eksperthinnangu ning K-keskmise klasterdamise alusel
diferentseeritud okas- ja lehtpuu alad. Järgnenud eeltöötlemise sammudes laeti
alla sobivad Sentinel-2 Level-2A tooted, teostati atmosfääriline korrektsioon
ning genereeriti georefereeritud GeoTIFF rasterfragmendid.

Seejärel peenhäälestati ja võrreldi kolme tipptasemel segmentatsiooniraamistiku
(DINO v2 + SAM v2, U-Net, Segmentation Model Zoo modulaarpead: DeepLabv3, FPN)
tulemuslikkust, mõõdetuna peamiselt IoU ja Dice koefitsiendiga. Tulemused
näitasid, et DINO v2 baasil ühendatud väheste näidete treeningu skeem
pakkus kõige paremat segmenteerimistäpsust. U-Net jäi veidi alla, kuid
kergesti peenhäälestatavate hüperparameetrite tõttu osutus produktiivseks
lähtepunktiks.

Uurimistöö tõestas, et isejuhitud peaahelad suudavad väheste andmetega anda
konkurentsivõimelisi tulemeid satelliidipildisegmenteerimise valdkonnas,
vähendades sõltuvust suurest märgendatud andmestikust.
Samas toodi välja mitmed potentsiaalsed suunad edasiseks arenduseks, sh
super-resolutsiooni rakendamine, täiendavate spektraalribade ja
kommerts-satelliitide andmete kaasamine, ajarealise analüüsi süvendamine ning
pilve- ja varju maskimise strateegiate täiustamine.

Kokkuvõttes annab töö aluse automaatsete algoritmide kasutusele metsaraiete
täpsemaks jälgimiseks ja metsade jätkusuutlikuks haldamiseks. Edasised uurimused
võiksid laiendada metoodikat globaalsesse konteksti ja integreerida
mitmeallikalisi andmeid, et saavutada veelgi detailsem ja vastupidavam
metsanduse seire- ning valduse haldamise lahendus.
