Käesolev peatükk analüüsib DINO v2 mudeli võrdlust teiste tipptasemel mudelitega, eriti semantilise segmentimise kontekstis, ning põhjendab selle valikut metsaalade ja metsaraie tuvastamiseks satelliidipiltidelt.

Ühes uuringus võrreldi DINO v2 mudelit mitmete teiste segmenteerimismeetoditega
 geoloogiliste kompuutertomograafia (KT) skaneeringute analüüsil. Hinnati Otsu
 lävendamist, klastritehnikaid (K-keskmised, fuzzy C-means), juhendatud
 masinõppe meetodit (Random Forest) ja süvaõppe mudeleid (UNet, ResNet152 ja
 DINOv2). Tulemused näitasid, et eelnevalt treenitud DINO v2 demonstreeris
 tugevat jõudlust kivimipiltide klassifitseerimisel isegi siis, kui
 KT-skaneeringud ei kuulunud selle esialgsesse treeningandmestikku.
 Segmenteerimise ülesannetes ületas eriti LoRA abil peenhäälestatud DINO v2
 teisi meetodeid mitmeklassiliste ülesannete puhul, isegi piiratud andmetega.
 Visuaalne kontroll näitas, et DINO v2 poolt genereeritud segmenteerimismärgid
 olid sageli täpsemad kui algsed sihtmärgid. \cite{DINOv2RocksGeological}

Teises uuringus hinnati DINO v2 jõudlust vasaku koja segmenteerimisel MRI
 piltidelt. Eelnevalt treenitud DINO v2 saavutas Dice'i sarnasuse koefitsiendi
 (DSC)  87.1 \%. Võrdluses nullist treenitud U-Net
 mudeliga näitasid tulemused, et iseseisev õpe DINO v2 abil võib oluliselt
 parandada meditsiinilise pildisegmentimise tõhusust ja täpsust, vähendades
 vajadust suurte märgistatud andmestike järele. \cite{kunduAssessingPerformanceDINOv22024}

Lisaks on Panopticon , DINO v2 raamistikule ehitatud mudel, saavutanud
 tipptasemel tulemusi GEO-Bench võrdlustestis maakatte klassifitseerimisel,
 ületades teisi sensoragnostilisi mudeleid ning näidates konkurentsivõimet ka
 kindlatele sensoritele kohandatud mudelitega. See demonstreerib DINO v2
 arhitektuuri tugevust ja kohanemisvõimet erinevate kaugseire andmetüüpidega. \cite{PanopticonAdvancingAnySensor}

Ühes teises töös võrreldi DINO (DINO v2 eelkäija) laiendust DINO-MC teiste
iseseisvalt õppivate meetoditega kaugseire ülesannetes. Lineaarse testimise
käigus saavutas DINO-MC 2.56\% kõrgema täpsuse väiksema eelnevalt treenitud
andmestikuga kui SeCo. Peenhäälestamisel kahel kaugseire klassifitseerimise ja
muutuste tuvastamise ülesandel ületas DINO-MC nii DINO kui ka SeCo mudeleid. \cite{ExtendingGloballocalView}

Ülaltoodud näited illustreerivad DINO v2 mudeli tugevust ja mitmekülgsust
erinevates segmentimise ülesannetes, sealhulgas geoloogiliste, meditsiiniliste
ja kaugseire piltide analüüsil. Mudeli võime saavutada konkurentsivõimelisi
tulemusi ilma spetsiifilise peenhäälestuseta ning selle parem üldistusvõime
muudavad selle sobivaks valikuks metsaalade ja metsaraie tuvastamiseks
satelliidipiltidelt.