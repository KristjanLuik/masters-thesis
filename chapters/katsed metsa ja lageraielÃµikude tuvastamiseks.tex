\textbf{Kodeerijad, Optimeerijad ja Segmentatsioonipead}

Uurimisöö põhiosa moodustavad DINO v2 ja SAM v2 segmentatsioonimudelite eksperimendid erinevad üksteisest nii mudeli
arhitektuuri, hüperparameetrite otsingu kui ka pea-(head-)komponentide poolest,
võimaldades võrrelda segmentatsiooni-võimekust erinevates seadistustes.

DINO v2 katse käivitamisel imporditakse esmalt eeltreenitud ViT-selgroo (backbone)
DINO v2 mudelist ja initsialiseeritakse uus DinoV2Segmenter, mille
dinov2\_backbone\_name-parameetriga valitakse ViT-versioon (näiteks ``small''
või ``base''). Selgroo kaalusid on võimalik kas täielikult külmutada või
lasta neil treeningus uuesti kohanduda (freeze\_backbone=True/False), mis
võimaldab hinnata nii puhtalt eelõpetatud omaduste kui ka peenhäälestamisel põhineva lähenemise
mõju. Peade valikus on kolm erinevat varianti: lihtsat LinearClassifier-pead,
hierarhiline UPerNetHead-lahendus ning FPN-pea (FPNHead), mis kombineerib
eri-tasandilisi omaduskaarte püramiidi kujul. Igal peal on omapärane viis,
kuidas manuspaigalt (patch-embedding) kogutud omadused ümber vormida lõplikuks
klassi-logititeks või täpsemaks semantilise ruumi rekonstruerimiseks.
Treeningprotsessis kasutatakse optimeerijana AdamW-i ja kadufunktsioonina mudeli
oma sisend-väljastuse kaotusmõõdikut (outputs.loss), samas kui Optuna
hüperotsing kontrollib erinevaid õppemäära, selgroo külmutamise ja pea tüübi
kombinatsioone. Kõik eksperimendid salvestavad iga tsükli järel nii IoU kui ka
Dice tulemused ning parimad kontrollpunktid kopeeritakse lõppanalüüsiks.
\bigskip

\begin{longtable}{lllll}
        \textbf{Tunnus} & \textbf{Backbone} & \textbf{Freeze Backbone} & \textbf{Head Type}  & \text{Optimeerija} \\
        \hline
        Trial 1 & ViT-small & Yes                    & LinearClassifier  & AdamW \\
        Trial 2 & ViT-small & Yes                    & UPerNetHead       & AdamW \\
        Trial 3 & ViT-small & Yes                    & FPNHead           & AdamW \\
        Trial 4 & ViT-small & No                     & LinearClassifier  & AdamW \\
        Trial 5 & ViT-small & No                     & UPerNetHead       & AdamW \\
        Trial 6 & ViT-small & No                     & FPNHead           & AdamW \\
        Trial 7 & ViT-base  & Yes                    & LinearClassifier  & AdamW \\
        Trial 8 & ViT-base  & Yes                    & UPerNetHead       & AdamW \\
        Trial 9 & ViT-base  & Yes                    & FPNHead           & AdamW \\
        Trial 10& ViT-base  & No                     & LinearClassifier  & AdamW \\
        Trial 11& ViT-base  & No                     & UPerNetHead       & AdamW \\
        Trial 12& ViT-base  & No                     & FPNHead           & AdamW \\
        \hline
    \caption{DINO v2 eksperimendi seadistused}
    \label{tab:dinov2_experiments}
\end{longtable}


SAM v2 katsed on üles ehitatud Meta ``Segment Anything Model'' teise põlvkonna
raamistiku põhjal ja kasutavad YAML-konfiguratsiooni, et deklareerida nii
SamConfig parameetrid kui ka SamModel tüüp. Eksperimendi konfiguratsioonis saab
samuti määrata selgroo külmutamise taseme, et võrrelda, kas pildipõhised
tunnused vajavad täiendavat peenhäälestust. Andmestiku laadimine põhineb .
pt-failidest kokku kogutud Sentinel-andmetel, mida töödeldakse
CustomForestryDataset klassi kaudu, et rakendada sobivad pilditransformatsioonid
(skaleerimine, normaliseerimine jne). Treening- ja valideerimislaadurid töötavad
batch\_size ning varisemise (drop\_last) parameetrite alusel, seejärel viiakse
läbi epohhipõhine treening ja valideerimine, kasutades train\_one\_epoch ja
validate\_one\_epoch funktsioone. Kadufunktsiooniks on klassikaline
CrossEntropyLoss ning optimeerijaks samuti AdamW. FPN-pea lisamine SAM
eksperimendis võimaldab sügavamaid võrdlusi, sest nii DINO v2 kui ka SAM v2
mudelid kasutavad nüüd sama püramiidipõhist pea-arhitektuuri, mis toetab
peenemat lõpliku segmentatsiooni juhtimist. Lõplik analüüs koondab tulemused
mudeli konfiguratsioonide, pea-valiku ja tuumvõrgu külmutamise
kombinatsioonides, et tuvastada kõige tõhusam lähenemine metsa alade ja
lageraielõikude tuvastamiseks satelliitpiltidelt.

\bigskip
\begin{longtable}{lllll}
    \textbf{Tunnus} & \textbf{Model Config} & \textbf{Freeze Backbone} & \textbf{Head Type} & \textbf{Optimeerija} \\
    \hline
        Trial 1 & default      & Yes                    & MaskDecoder (orig.) & AdamW \\
        Trial 2 & default      & Yes                    & FPNHead             & AdamW \\
        Trial 3 & default      & No                     & MaskDecoder (orig.) & AdamW \\
        Trial 4 & default      & No                     & FPNHead             & AdamW \\
        Trial 5 & large        & Yes                    & MaskDecoder (orig.) & AdamW \\
        Trial 6 & large        & Yes                    & FPNHead             & AdamW \\
        Trial 7 & large        & No                     & MaskDecoder (orig.) & AdamW \\
        Trial 8 & large        & No                     & FPNHead             & AdamW \\
    \hline
    \caption{SAM v2 eksperimendi seadistused}
    \label{tab:samv2_experiments}
\end{longtable}
